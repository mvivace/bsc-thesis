\chapter{Il modello a tre fattori di Eugene Fama e Kenneth French}
\label{chaptre}

\section{Introduzione}


Studi successivi al Capital Asset Pricing Model e all'Arbitrage Pricing Theory mostrano una correlazione fra i rendimenti medi delle azioni ordinarie e le caratteristiche specifiche dell'impresa misurate attraverso indici come la capitalizzazione di mercato, il livello di indebitamento (Leverage), l'Earning to Price (E/P) e il Book to Market (BE/ME) \cite{fama_common_1993}.

Banz (1981) aveva scoperto che i titoli a bassa capitalizzazione presentavano rendimenti medi maggiori rispetto ai titoli ad alta capitalizzazione, Bhandari (1988) aveva individuato una relazione positiva fra il livello di indebitamento e i rendimenti medi dei titoli. Questo comportamento può essere giustificato dal fatto che a livelli di indebitamento maggiori corrispondono rischi aggiuntivi. 
Mentre Stattman (1980) e Rosenberg, Reid  e Lanstein  (1985) avevano individuato una connessione fra i rendimenti medi dei titoli statunitensi e l'indice Book to Market.

Infine Ball (1978) e Basu (1983) ritenevano che anche l'indice Earning to Price potesse fornire una spiegazione dei rendimenti medi. Alti valori dell'indice sarebbero legati a un alto livello di rischio e ad alti profitti. 

Poiché questi comportamenti non vengono spiegati dal CAPM vengono tipicamente definiti “anomalie”. 

A dimostrazione di ciò, da una analisi dei rendimenti medi relativi al periodo precedente al 1969 effettuata da alcuni economisti come Black, Jensen e Scholes (1972) e Fama e MacBeth (1973), emerge una relazione positiva fra i rendimenti e il fattore $\beta_M$. Tuttavia i test effettuati negli anni successivi da Fama e French e da altri studiosi come Reinganum (1981), Lakonishok e Shapiro (1986), mostrano che tale relazione scompare durante il periodo tra il 1963 e il 1990.

Nei propri scritti \textit{Common risk factors in the returns on stocks and bonds} (1993) \cite{fama_common_1993} e \textit{Multifactor Explanations of Asset Pricing Anomalies} (1996) \cite{fama_multifactor_1996} Eugene Fama e Kenneth French propongono una strategia di pricing alternativa in grado di catturare gran parte delle anomalie presentate. Il modello prende il nome di \textit{Three Factors Model} e descrive i rendimenti dei portafogli in funzione di 3 fattori di rischio: il fattore di rischio sistematico misurato da $\beta_M$, il fattore dimensionale e l'indice Book to Market. 

  

\section{Indici di bilancio}

Come accennato, gli studi sui rendimenti medi dei titoli dimostrano che le caratteristiche di un'impresa, misurate attraverso gli indici di bilancio, sono particolarmente rilevanti nel processo di pricing dei titoli. Per tale ragione prima di proseguire con l'analisi del modello è importante comprenderne il significato.

Gli \textit{indici di bilancio}, sono dei valori che si ricavano dal rapporto fra voci dello stato patrimoniale e/o del conto economico. Lo studio e la comparazione di questi indici consentono di ottenere informazioni più specifiche sulla performance economico-finanziaria e sulla situazione patrimoniale di un'impresa.

\paragraph{La capitalizzazione di mercato (ME)}

La capitalizzazione di mercato rappresenta il valore di mercato di un'impresa quotata ed è data dal prodotto fra il numero di azioni in circolazione e il prezzo di quotazione dei titoli. 

\`E uno strumento di valutazione utilizzato dagli investitori che consente di stimare non soltanto le dimensioni della società ma anche gli obiettivi e il profilo di rischio dei titoli. 

Inoltre, fornisce una misura del valore attuale dell'impresa; infatti i prezzi, fluttuando a causa della domanda e dell'offerta dei titoli, comunicano le aspettative degli investitori su una determinata società. Se per esempio vi sono aspettative di crescita la domanda dei titoli aumenta e il prezzo cresce, viceversa i venditori offriranno in massa i titoli facendo crollare i prezzi. 

\`E possibile distinguere 3 tipi di società: 
\begin{description}
	\item[Large Cap] sono caratterizzate da una capitalizzazione di mercato pari a 10 miliardi o più. Sono generalmente di grandi dimensioni e per tale ragione ritenute più stabili, tuttavia non offrono le stesse opportunità di crescita delle società a media o bassa capitalizzazione.
	\item[Mid Cap] sono caratterizzate da una capitalizzazione di mercato tra i 2 e i 10 miliardi. Tali società sono  in fase di espansione pertanto caratterizzate da aspettative di crescita molto rapida. Non essendo del tutto stabili sul mercato ad esse sono associati dei rischi maggiori rispetto alle large cap.
	\item[Small Cap]sono caratterizzate da una capitalizzazione di mercato tra i 300 milioni e i 2 miliardi. Tali compagnie sono molto giovani e ad esse sono associate grandi aspettative di crescita soprattutto se collocate in settori emergenti. Le piccole dimensioni consentono una crescita rapida in periodi di prosperità economica, tuttavia potrebbero presentare livelli di indebitamento molto alti, il che le renderebbe più fragili nelle fasi di recessione economica. 
\end{description}

\paragraph{Leverage}

Nell'analisi di bilancio, il leverage rappresenta il livello di indebitamento ovvero la misura in cui una società ricorre a capitale di terzi per finanziarsi. La formula per calcolare l'indice è la seguente: 
\begin{equation}
\begin{split}
L&=\frac{Totale\hspace{0,1cm}fonti\hspace{0,1cm}di\hspace{0,1cm}finanziamento}{Capitale\hspace{0,2cm}proprio}\\
&= \frac{Capitale\hspace{0,1cm} di\hspace{0,1cm} debito + Capitale\hspace{0,2cm}proprio}{Capitale\hspace{0,2cm}proprio}
\end{split}
\end{equation}

Dove le fonti di finanziamento rappresentano le risorse necessarie per lo svolgimento dell'attività. Possono essere interne o esterne, nel primo caso si parla di capitale proprio o di rischio  e viene apportato dall'imprenditore e/o dai soci dell'impresa, nel secondo caso si parla di capitale di debito e rappresenta un prestito fornito da soggetti terzi sul quale viene pagato un tasso di interesse.

Se l'indice assume un valore pari a 1 significa che i finanziamenti sono tutti rappresentati da capitale proprio, se invece assume un valore fra 1 e 2 il capitale proprio è maggiore rispetto al livello di indebitamento, infine se assume un valore superiore a 2 i debiti sono superiori al capitale proprio. Un valore molto basso del rapporto indica un'elevata indipendenza finanziaria viceversa un valore elevato evidenzia un'eccessiva dipendenza dai mezzi di terzi e potrebbe essere un segnale di sotto capitalizzazione, i mezzi di terzi non sono adeguatamente controbilanciati dai mezzi propri e la situazione finanziaria della società potrebbe essere compromessa.

\paragraph{Earning to Price (E/P)}Questo indice è dato dal rapporto fra gli utili per azione (EPS) e il prezzo per azione. 

Tanto più il rapporto è basso tanto maggiori sono le aspettative di crescita sulla società e tanto maggiore sarà il prezzo che un investitore sarà disposto a pagare per gli utili attesi. Viceversa un valore alto potrebbe indicare scarse opportunità di crescita pertanto il prezzo che gli investitori saranno disposti ad offrire sarà inferiore. 

Questo indice consente agli investitori di determinare il valore di una società ed eventualmente di confrontarlo con società del medesimo settore, inoltre è un valido strumento per determinare se una società è attualmente sopravvalutata (basso E/P) o sottovalutata (alto E/P) rispetto agli andamenti passati.

\paragraph{Book to Market (BE/ME)}  

Infine l'indice Book to Market è definito come rapporto fra valore contabile (\textit{BE}) e valore di mercato di una società \textit{(ME)}. Il valore contabile viene calcolato sottraendo alle attività iscritte a bilancio le passività e il capitale intangibile. Tale valore potrebbe essere superiore o inferiore al valore di mercato. Gli investitori si avvalgono di questo indice per valutare se una società è sopravalutata o sottovalutata. Una società che presenta un indice superiore a 1 viene definita \textit{value stock}, il valore contabile è superiore al valore di mercato quindi potrebbe essere sottovalutata, tali società risulteranno attraenti per gli investitori perché sarà possibile comprare i titoli ad un costo inferiore tuttavia sono ritenute più rischiose. Un valore inferiore a 1 identifica invece una società sopravalutata \textit{growth stock}, in tal caso in proiezione di profitti futuri gli investitori saranno disposti ad offrire un prezzo superiore rispetto al valore contabile della società, tali società presentano una capitalizzazione di mercato superiore e pertanto sono ritenute meno rischiose.

\section{Anomalie di mercato}

Una delle prime anomalie emerse nel mercato statunitense venne riscontrata da Banz nel 1981 \cite{banz_relationship_1981}, i risultati dei suoi studi sulle azioni ordinarie del NYSE mostrarono che tra il 1936 e il 1975 le Small Cap presentavano rendimenti medi superiori rispetto a quelli delle Large Cap. Tale comportamento prese il nome di \textit{size effect}. 

Ad oggi non vi sono delle teorie comprovate che riescano a spiegare tale fenomeno. Una delle spiegazioni plausibili venne data da Klein e Bawa (1977), i quali mostrarono che le azioni delle società sulle quali si avevano poche informazioni a disposizione offrivano rendimenti maggiori rispetto alle società di cui si disponevano grandi quantità di informazioni. 

Infatti, solo qualora il mercato abbia sufficienti informazioni riuscirebbe a prezzare correttamente il titolo, viceversa il rischio di valutare erroneamente i titoli sarebbe maggiore. In tal caso l'investitore richiederebbe un premio supplementare per compensare l’assenza di informazioni. Questo fenomeno viene definito \textit{neglected firm effect}. 

Mentre Basu (1977, 1983) fu il primo a documentare il \textit{E/P effect},  secondo cui i rendimenti delle azioni ordinarie del NYSE sono influenzati dall'andamento degli utili \cite{basu_investment_1977} \cite{basu_relationship_1983}. Le analisi svolte tra il 1963 e il 1979, prendendo in considerazione sia gli utili che la dimensione della società, mostrarono che le azioni ordinarie con alti valori di E/P presentavano rendimenti medi aggiustati per il rischio superiori a quelli delle azioni con bassi valori di E/P. I risultati indicavano inoltre che, anche se i rendimenti delle Small Cap erano superiori rispetto a quelli delle Large Cap, l'effetto size scompariva nel momento in cui le azioni venivano differenziate sulla base del valore E/P. Da ciò l'autore ne conclude che la dimensione misurata dalla capitalizzazione di mercato potrebbe rappresentare una buona proxy anche per il E/P effect ma non viceversa. 

Il perdurare di tale anomalia, potrebbe essere in linea con l'ipotesi di inefficienza dei mercati e di un'errata costruzione del CAPM dovuta all'omissione di fattori di rischio che pertanto lo rendono inadeguato.

De Bondt e Thaler (1985) \cite{de_bondt_does_1985} forniscono una spiegazione comportamentale a questo fenomeno. Secondo una loro analisi, gli investitori tendono a reagire in modo eccessivo a notizie inaspettate o drammatiche sui profitti (\textit{overreaction}). Per esempio società con alti valori di E/P potrebbero essere temporaneamente sottostimate a causa di una eccessiva reazione a notizie sfavorevoli, tuttavia, una volta che gli utili si rivelano migliori rispetto alle aspettative, i prezzi tendono ad aumentare e viceversa. La presenza di overreaction genera fenomeni di inversione dei prezzi che causano rendimenti anomali. 

Infine l'ultima anomalia è quella relativa all'indice Book to Market, nello specifico Rosenberg, Reid  e Lanstein  (1985) e successivamente Fama e French (1992) dimostrarono una tendenza delle società con alti valori dell'indice (value stock) a presentare rendimenti superiori rispetto alle società con bassi valori (growth stock). Vi sono diverse interpretazioni a riguardo: alcuni economisti ritengono che ciò dipenda da una sopravvalutazione o sottovalutazione della società, altri ritengono che l'indice BE/ME sia una misura del rischio, pertanto i rendimenti offerti dalle società value, che sono ritenute più rischiose, dovranno essere maggiori rispetto alle società growth.


\section{Regressione cross-section dei rendimenti attesi delle azioni di Fama e French} \label{secdue}

Sulla base delle anomalie rilevate in precedenza e utilizzando il medesimo approccio messo in atto da Fama e Macbeth (1973), l'analisi di regressione cross-section, nel 1992 Fama e French propongono uno studio, il cui obiettivo è quello di mostrare l'influenza del mercato $\beta_M$, della dimensione, dell'indice E/P, del leverage e del Book to Market sui rendimenti medi dei titoli \cite{fama_cross-section_1992}. 

L'\textit{analisi cross-section} consiste nello studio dei dati raccolti (ad esempio altezza, rendimenti, ecc.) su un gruppo di soggetti nella medesima unità temporale. L'obiettivo è quello di confrontare i dati dei diversi soggetti. Si contrappone all'\textit{analisi time-series} poiché quest'ultima viene effettuata sui dati raccolti in diverse unità temporali, l'obiettivo è un confronto fra i valori assunti in tempi diversi dalla popolazione oggetto di studio \cite{noauthor_time_2019}. 

La regressione lineare consiste nell'individuare la relazione che esiste fra una o più variabili indipendenti e una variabile dipendente. Tale relazione può essere espressa come segue: 
\begin{equation}
y=b_0+b_1X_1+b_2X_2+......+b_NX_N
\end{equation}
dove $y$ è definita variabile dipendente, $X=1,2,...,N$ sono le variabili indipendenti e $b= 0,1,2,.....,N$ rappresentano i coefficienti di regressione. 


Nello specifico, l'analisi cross-section di Fama e MacBeth è un pratico strumento che consente di spiegare in che modo i fattori di rischio presi in esame influenzano e determinano il rendimento di un titolo o di un portafoglio. L'obiettivo di questa strategia è definire il premio per il rischio offerto agli investitori per l'esposizione a tali fattori. 

\begin{description}
	\item[Step 1.] Attraverso la regressione time-series dei rendimenti rispetto ad uno o più fattori di rischio vengono stimati i coefficienti di regressione $\beta$ che in questo caso specifico rappresentano la sensibilità dei titoli ai fattori di rischio. 
	
	Per esempio consideriamo i rendimenti $R_{i}$ dei titoli $i=1,2,....,n$ rilevati in un arco temporale $t$ con $t= 1,2,....,T$ e fattori di rischio  $F_m$ tali che $m=1,2,...M$. 
	
	La regressione dei rendimenti dei titoli effettuata per ogni periodo $t$ permette di stimare i coefficienti $\beta$:
\begin{equation}
\begin{split}
	R_{1,t}&= \alpha_1 + \beta_{1,F1}F_{1,t}+\beta_{1,F2}F_{2,t}+.....+\beta_{1,FM}F_{M,t}+ \epsilon_{1,t}\\
		R_{2,t} &= \alpha_2 + \beta_{2,F1}F_{1,t}+\beta_{2,F2}F_{2,t}+.....+\beta_{2,FM}F_{M,t}+ \epsilon_{2,t}\\
		\vdots &  \\
			R_{n,t}&= \alpha_n + \beta_{n,F1}F_{1,t}+\beta_{n,F2}F_{2,t}+.....+\beta_{n,FM}F_{M,t}+ \epsilon_{n,t}
	\end{split}
\end{equation}
	\item [Step 2.] Per ogni titolo $i$ e per ogni periodo $t$ viene poi effettuata una regressione cross-section dei rendimenti utilizzando i coefficienti $\beta$ calcolati nel primo step. Questa procedura ci consente di misurare il premio per il rischio $\gamma_{m,t}$ relativo ad ogni fattore $m$ e ad ogni periodo $t$. 
	\begin{equation}
	\begin{split}
	R_{i,1}= & \gamma_{t,0}+\gamma_{t,1}\beta_{t,1}+\gamma_{t,2}\beta_{t,2}+.....+\gamma_{t,M}\beta_{t,M} + \epsilon_{i,t}\\
		R_{i,2}= & \gamma_{t,0}+\gamma_{t,1}\beta_{t,1}+\gamma_{t,2}\beta_{t,2}+.....+\gamma_{t,M}\beta_{t,M} + \epsilon_{i,t} \\
\vdots &  \\
R_{i,T} =& \gamma_{t,0}+\gamma_{t,1}\beta_{t,1}+\gamma_{t,2}\beta_{t,2}+.....+\gamma_{t,M}\beta_{t,M} + \epsilon_{i,t}
	\end{split}
	\end{equation}
\end{description}




\subsection{Dati}
Nell'analisi di Fama e French vengono considerati i rendimenti delle società non finanziarie presenti nel NASDAQ, AMEX e NYSE dal 1962 al 1989. 

I valori delle variabili prese in esame vengono rilevati nel periodo $t-1$ mentre i rendimenti vengono presi in considerazione da Luglio del periodo $t$ fino a Giugno del periodo $t+1$. Questa procedura viene messa in atto per assicurare che le variabili utilizzate per spiegare i rendimenti siano note prima dei rendimenti stessi. 

Inizialmente i titoli delle società oggetto dell'analisi vengono suddivise in 10 portafogli in base alla capitalizzazione di mercato, ognuno di essi viene ulteriormente suddiviso in  10 portafogli che identificano valori di $\beta$ \textit{pre ranking} differenti. I $\beta$ \textit{pre ranking} vengono stimati attraverso i rendimenti passati. In questo modo è possibile costruire 100 portafogli che definiremo $\beta-size$ e calcolare per ognuno di essi i rendimenti mensili da Luglio a Giugno dell'anno successivo per un periodo compreso fra il 1969 e il 1990, questi rendimenti vengono definiti post-ranking.

Attraverso la regressione dei rendimenti post ranking dei portafogli (step 1) sarà possibile calcolare i beta post ranking. Questi ultimi vengono poi assegnati ai singoli titoli del portafoglio.

\subsection{$\beta$ e Size}

Passando alle stime, la \ref{fig:table-i-boh} mostra che la costruzione dei portafogli $\beta-size$ rispetto a quella basata sulla sola dimensione (ME) consente di ampliare il range dei valori di $\beta$. 

Tra i 10 portafogli (colonna 1) i valori di  $\beta$ variano da 1.44 (basso ME) a 0.92 (alto ME). Tra i 100 portafogli il $\beta$ assume valori fra 0.53 a 1.79.


\begin{figure}[h]
	\centering
	\includegraphics[width=0.9\linewidth]{"imgs/table I boh"}
	\caption{Portafogli $\beta-size$}
	\label{fig:table-i-boh}
\end{figure}
	
Inoltre Fama e French affermano che i portafogli costruiti solo sulla dimensione (ME) mostrano una relazione positiva fra rendimenti e $\beta$. Tuttavia tale relazione perfettamente positiva vi è anche tra il fattore size e $\beta$. Pertanto non è possibile distinguere l'effetto size dall'effetto $\beta$ sui rendimenti. 

Attraverso la suddivisione dei portafogli size in base ai valori dei $\beta$ pre ranking si nota una forte relazione fra rendimenti e il fattore size mentre scompare la relazione positiva fra rendimenti e $\beta$.

La seguenti figure mostrano i rendimenti medi post-ranking dei portafogli costruiti o sulla dimensione \ref{fig:table-ii-part-1} o sui $\beta$ \ref{fig:table-ii-part-2}. 
\newpage

\begin{figure}[h]
	\centering
\includegraphics[width=1\linewidth]{"imgs/table II part 1"}
\caption{Portafoglio formato sulla dimensione}
\label{fig:table-ii-part-1}
\end{figure}



\begin{figure}[h]
	
	\centering
\includegraphics[width=1\linewidth]{"imgs/TABBLE II PART 2"}
\caption{Portafoglio formato sui $\beta$}
\label{fig:table-ii-part-2}
\end{figure}


Come possiamo notare nella \ref{fig:table-ii-part-1} vi è una forte relazione negativa fra i rendimenti medi e il fattore size che conferma l'iniziale intuizione di Banz del 1981. Il rendimento associato al portafoglio 1A è pari a 1.64 e tale valore decresce all'aumentare della dimensione fino ad arrivare a 0.90 per il portafoglio 10B. 

In aggiunta possiamo identificare una forte relazione positiva fra rendimenti medi e $\beta$. Questo caso sembra supportare le implicazione del CAPM. Tuttavia questo può essere associato ad una stretta correlazione fra il fattore size e il fattore $\beta$.


Nella \ref{fig:table-ii-part-2} possiamo vedere che rispetto al caso precedente i valori di beta variano in un range molto più ampio da 0.79 a 1.63, inoltre al variare di $\beta$ i rendimenti subiscono solo piccole variazioni, pertanto non vi è una chiara relazione tra i due. Per esempio se guardiamo i rendimenti dei due portafogli estremi 1A e 10B nonostante i valori di beta siano molto diversi, i rispettivi rendimenti sono pari a 1,20 e 1,18. Questo conferma quanto affermato da Reinganum nel 1981 cioè un'assenza di relazione fra rendimenti e fattore $\beta$ tra il 1964 e il 1979. 

Una possibile spiegazione a tale fenomeno viene data dai due autori, i quali ritengono che tra i beta e le altre variabili esplicative possa esserci una correlazione e questo potrebbe oscurare la relazione fra rendimenti e beta. Tuttavia questa spiegazione non è sufficiente a chiarire il motivo per cui anche quando utilizzati singolarmente i fattori beta non spiegano i rendimenti dei titoli. 

\subsection{E/P, Book to Market e Leverage}

Le seguenti tabelle mostrano i rendimenti medi dei portafogli costruiti utilizzando l'indice Book to Market e l'Earning to Price:
\begin{figure} [h]
	\centering
	\includegraphics[width=1\linewidth]{"imgs/table IV part 1"}
	\caption{Titoli ordinati in base all'indice Book to Market}
	\label{fig:table-iv-part-1}
\end{figure}

\begin{figure} [h]
	\centering
\includegraphics[width=1\linewidth]{"imgs/table iv part 2"}
\caption{Titoli ordinati in base all'indice Earning-Price}
\label{fig:table-iv-part-2}
\end{figure}



Anche in questo caso i risultati dimostrano le evidenze emerse in precedenza da altri autori. 

\paragraph{Book to Market}
Nella \ref{fig:table-iv-part-1} possiamo vedere una relazione fortemente positiva tra i rendimenti medi e i valori del Book to Market, i rendimenti medi crescono da 0.30 per il portafoglio 1A con il minor valore dell'indice fino ad arrivare a 1.83 per i portafogli con i più alti valori dell'indice. Con uno spread pari a 1.53. Tale differenza è ben due volte superiore alla differenza dei rendimenti medi dei portafogli (1A, 10B) costruiti in base al fattore dimensione pari a 0.74 (1.64-0.90) nella \ref{fig:table-ii-part-1}


Alti valori BE/ME derivano da scarse prospettive di guadagno che inducono gli investitori a sottovalutare la società. Pertanto gli alti rendimenti medi di portafogli con alti valori di BE/ME sono consistenti con l'ipotesi che il BE/ME riesca a catturare le variazioni dei rendimenti medi dovute ad una situazione di difficoltà (\textit{distress}). 

Infine è poco probabile che la relazione fra i rendimenti e l'indice BE/ME sia legata a un effetto dei $\beta$ post ranking, infatti i $\beta$ subiscono solo una leggera variazione fra i diversi portafogli. Se guardiamo i due portafogli estremi (1A e 10B) notiamo una variazione dei rendimenti da 0.30 a 1.83 tuttavia i loro $\beta$ sono molto simili (1.36 e 1.35), pertanto non è possibile giustificare la variazione dei rendimenti con i $\beta$. 

\paragraph{Earning/Price}Nella \ref{fig:table-iv-part-2} possiamo notare che all'aumentare dell'Earning to Price ratio vi è una crescita dei rendimenti medi. Vediamo infatti che i rendimenti medi decrescono da 1.46 per valori negativi di E/P fino a 0.93 per portafogli con valori bassi ma positivi di E/P per poi crescere ulteriormente fino a 1.72 all'aumentare dell'indice.

Come accennato, nel 1978 Ball aveva intuito che poiché i profitti correnti erano in grado di approssimare quelli futuri l'indice E/P poteva rappresentare una buona proxy per spiegare i rendimenti attesi. Ciò è verificato solo per valori dei profitti positivi.

Le figure \ref{fig:table-iv-part-1} e la \ref{fig:table-iv-part-2} mostrano che aggiungendo la dimensione della società alla regressione viene eliminata la capacità dell'E/P (per i valori positivi). Si ricava che gran parte della relazione che sussiste tra l'indice E/P e i rendimenti medi si deve ad una correlazione positiva fra l'indice E/P e il BE/ME. Società con alti valori di E/P sono caratterizzati da alti valori del BE/ME e viceversa. Pertanto è possibile spiegare i rendimenti medi ricorrendo all'indice BE/ME.

\paragraph{Leverage}Infine per ciò che concerne il leverage, se prendiamo in considerazione due misure di indebitamento cioè \textit{market leverage} ln(A/ME) che viene calcolato come rapporto fra il valore contabile e il valore di mercato di un'attività e il \textit{book leverage} ln(A/BE) che è il rapporto fra il valore contabile dell'attività e il valore contabile delle azioni ordinarie possiamo vedere che queste due misure influenzano i rendimenti ma con segni opposti. 

Come in Bhandari (1988) ad un alto valore di market leverage sono associati alti rendimenti e le pendenze sono sempre positive, mentre ad alti valori del book leverage sono associati bassi rendimenti e le pendenze sono sempre negative. 

Attraverso queste due misure è possibile arrivare ad una soluzione molto semplice, le pendenze delle due variabili in valore assoluto sono molto simili, pertanto i rendimenti medi possono essere spiegati dalla differenza fra queste due variabili, tuttavia la differenza fra i due fornisce esattamente l'indice BE/ME: ln(BE/ME)= ln(A/ME)-ln(A/BE). 

Il legame fra l'indice $BE/ME$ e le due misure del leverage ci suggeriscono che alti valori del $BE/ME$ sono dati da scarse prospettive di crescita della società  rispetto alle società con migliori prospettive e che pertanto presentano bassi valori di $BE/ME$.

In breve, l'indice $BE/ME$ che cattura una situazione di difficoltà può essere interpretato come un involontario effetto leverage che viene catturato dalla differenza fra market leverage e book leverage.  

Possiamo riassumere l'analisi effettuata come segue:
\begin{itemize}
	\item Se classifichiamo i portafogli in relazione ai valori di $\beta$ e non alla dimensione della società (ME), non viene riscontrata alcuna relazione tra $\beta$ e rendimenti;
	\item Il ruolo opposto del market leverage e del book leverage viene catturato dall'indice $BE/ME$;
	\item La relazione fra l'indice E/P e i rendimenti medi sembra essere assorbita congiuntamente dal fattore size e dall'indice BE/ME.  
\end{itemize}

\section{Modello a tre fattori} \label{sectre}

I risultati degli studi effettuati da Fama e French sui rendimenti delle azioni ordinarie tra il 1963 e il 1990 mostrano che il fattore dimensione (Size) e l'indice Book to Market sono in grado di catturare le variazioni cross-section dei rendimenti medi associati alla dimensione, all'indebitamento (leverage), all'indice BE/ME e E/P di una società. 

Questa analisi, esclude il ruolo del fattore $\beta$. Tuttavia gli stessi autori ritengono che qualora includessimo nell'analisi altri titoli diversi dalle azioni ordinarie come i bond e i bills, il ruolo del premio per il rischio di mercato sarebbe ben diverso e il coefficiente $\beta$ potrebbe assumere importanza. 

Inoltre, l'analisi effettuata permette di spiegare le differenze fra i rendimenti di diverse azioni attraverso valori diversi della dimensione e dell'indice BE/ME, tuttavia questi fattori non sono in grado di spiegare la differenza fra i rendimenti delle azioni e il titolo privo di rischio. Questo lavoro viene infatti svolto dal fattore di mercato. La regressione dei rendimenti rispetto ai tre fattori di rischio produce un coefficiente $\beta$ molto vicino a 1. Pertanto il premio per il rischio di mercato può spiegare la differenza fra i rendimenti medi di un titolo e il titolo risk-free. 

I risultati emersi consentirono a Fama e French di costruire un modello generale di pricing dei titoli in grado di catturare le anomalie del CAPM.

Per definire i premi per il rischio associati ai due nuovi fattori di rischio, la dimensione e l'indice BE/ME, i due autori costruiscono 6 portafogli. Utilizzando i titoli del NASDAQ, AMEX e NYSE costruiscono i primi 2 portafogli suddividendo i titoli in base alla dimensione (ME), quindi individuano portafogli Small e portafogli Big. Successivamente suddividono i medesimi titoli in 3 portafogli in base al valore dell'indice BE/ME (Low, Medium, High) \cite{fama_common_1993}. Infine intersecando i portafogli ME e i portafogli (BE/ME) individuano 6 portafogli: 
\begin{itemize}
	\item S/L è un portafoglio che contiente titoli small e con bassi valori BE/ME;
	\item S/M è un portafoglio che contiente titoli small e con medi valori BE/ME;
	\item S/H è un portafoglio che contiente titoli small e con alti valori BE/ME;
	\item B/L è un portafoglio che contiente titoli big  e con bassi valori BE/ME;
	\item B/M è un portafoglio che contiente titoli big  e con medi valori BE/ME;
	\item B/H è un portafoglio che contiente titoli big  e con alti valori BE/ME.
\end{itemize}

Attraverso questi portafogli i due autori costruiscono i premi per il rischio relativi al fattore size e all'indice BE/ME. 

\paragraph{SMB} Small Minus Big è un portafoglio costruito dai due autori in modo da imitare il premio per il rischio relativo alla dimensione. \`E costruito come differenza fra la media aritmetica dei rendimenti di 3 portafogli small (S/L, S/M,S/H) e i rendimenti di 3 portafogli big (B/L, B/M, B/H).

\paragraph{HML} High Minus Low è un portafoglio costruito dai due autori in modo da imitare il premio per il rischio relativo al'indice BE/ME. \`E costruito come differenza fra la media aritmetica dei rendimenti dei portafogli con alti valori dell'indice (S/H e B/H) e i rendimenti dei portafogli con bassi valori dell'indice (S/L e B/L).

\paragraph{Market} Infine il premio per il rischio di mercato è il rendimento in eccesso di un portafoglio value-weighted dei titoli contenuti nei 6 portafogli costruiti precedentemente rispetto al rendimento del titolo risk-free (one month bill rate).

\hfill

 Il modello prevede che il rendimento atteso di un portafoglio $E(R_i)$ in eccesso rispetto al risk-free rate $r_f$ possa essere spiegato attraverso la sensitività dei rendimenti a 3 fattori di rischio: i) il rendimento in eccesso di un portafoglio di mercato rispetto al risk-free rate $E(R_i)-r_f$ ii) la differenza fra il rendimento di un portafoglio di piccole dimensioni (small) e il rendimento di un portafoglio di grandi dimensioni (large), $SMB$ iii) la differenza tra il rendimento di un portafoglio con alti valori di BE/ME e il rendimento di un portafogli con bassi valori di BE/ME, $HML$. Nello specifico:

\begin{equation}
E(R_i)-r_f= \alpha_i + b_i[E(R_M)-r_f]+s_iE(SMB)+ h_iE(HML)
\end{equation}
Dove $[E(R_M)-r_f, E(SMB), E(HML)$ rappresentano i premi attesi e $b_i,s_i,h_i$ sono i fattori di sensitività che rappresentano i coefficienti di regressione della regressione time-series. 

Per testare i risultati ottenuti Fama e French costruiscono 25 portafogli sulla base del fattore dimensionale e dell'indice BE/ME. 

Questi portafogli confermano l'evidenza emersa dal lavoro del 1992 cioè una relazione negativa fra i rendimenti e il fattore dimensionale, a società small sono associati rendimenti maggiori rispetto alle società big e una forte relazione positiva fra i rendimenti e l'indice BE/ME, ad alti valori sono associati alti rendimenti e viceversa. 

Nel 1995 i due autori mostrano che l'indice BE/ME e il fattore di sensitività relativo ad esso sono delle buone proxy per i periodi di difficoltà. Società deboli con profitti bassi e persistenti tendono ad avere alti valori BE/ME e coefficienti positivi; società più solide con profitti alti e persistenti tendono ad avere bassi valori BE/ME e coefficienti negativi. L'utilizzo di HML per spiegare i rendimenti è in linea con l'evidenza di Chan e Chen (1991) che afferma l'esistenza di una correlazione nei rendimenti legata ai periodi di difficoltà che non viene spiegata dal rendimento di mercato. Similmente l'utilizzo del fattore SMB è in linea con l'evidenza di Huberman a Kandel (1987) di una correlazione tra i rendimenti e le piccole società che non viene catturata dai rendimenti di mercato.

Inoltre il coefficiente di determinazione $R^2$ che rappresenta la quota delle variazioni della variabile dipendente spiegata da una variazione delle variabili indipendenti, varia da 0.83 a 0.97. Se il coefficiente di determinazione è pari a 1 significa che la variazione delle variabili indipendenti spiega perfettamente la variazione subita dalla variabile dipendente. Gli alti valori di $R^2$ che derivano dall'applicazione del modello a tre fattori indicano che è possibile replicare i rendimenti dei portafogli impiegando questo modello. 


\section{Validità del modello a tre fattori}

Il modello a tre fattori sembra catturare gran parte delle variazioni dei rendimenti cross-section. In particolare riesce a catturare le anomalie del CAPM. Inoltre è un modello in linea con le implicazioni dell'Arbitrage Pricing Theory la quale sosteneva l'esistenza di un rischio multidimensionale che implicava l'esistenza di diversi fattori di rischio. 

Il modello è una buona descrizione dei rendimenti dei portafogli formati sulla base della dimensione e dell'indice BE/ME. 

Tuttavia uno dei difetti del modello sembra essere anche in questo caso la mole di informazioni necessarie per poterlo applicare. 
