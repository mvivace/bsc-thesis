\chapter{Conclusioni}

L'elaborato ripercorre l'evoluzione dei modelli di pricing a partire dagli anni cinquanta con la Moderna Teoria del Portafoglio di Harry Markowitz fino agli anni novanta con il Modello a tre fattori di Eugene Fama e Kenneth French.

Come abbiamo avuto modo di constatare il modello di Markowitz rappresenta senz'altro un enorme innovazione e fornisce strumenti validi che hanno permesso la nascita dei modelli di pricing successivi. Lo stesso CAPM fa propri questi concetti e fornisce una spiegazione del rendimento definendolo funzione lineare di un rischio sistematico o di mercato. Ad oggi rappresenta uno dei modelli maggiormente utilizzati e diffusi grazie alla sua semplicità applicativa nonostante presenti limiti importanti come assunzioni molto lontane dalla realtà (assenza di tasse, accesso alle informazioni a costo zero) e non includa fattori di rischio che dovrebbero essere presi in considerazione. 

Quest'ultimo incentiva diversi studiosi a cimentarsi in nuovi modelli che prendono il nome di modelli multifattoriali, in particolare abbiamo visto: l'Arbitrage Pricing Theory e il Modello a tre fattori di Fama e French. L'APT ripropone le medesime implicazioni del CAPM ma assumendo ipotesi meno stringenti e assumendo la presenza di più fattori di rischio fra i quali ritroviamo il fattore di rischio di mercato, consentendo il superamento dei limiti del CAPM.  Tuttavia, rappresenta un modello molto generale, infatti non vengono specificati nè la quantità nè la qualità dei fattori di rischio. 

Studi successivi mettono in luce una connessione fra le variabili microeconomiche e i rendimenti, Eugene Fama e Kenneth French ne individuano in particolare due (Size, Book/Market) e creano un modello che sembra catturare gran parte delle variazioni dei rendimenti cross-section e time-series, sottolineando in particolare una tendenza dei portafogli delle società poco capitalizzate e con alti valori dell'indice Book/Market a sovraperformare i portafogli di società maggiormente capitalizzate. 

I risultati dell'analisi svolta nel quarto capitolo confermano solo una parziale validità del modello; valori dell'intercetta diversi da zero suggeriscono infatti la presenza di fattori di rischio che non sono stati considerati nel modello, così come non vi è una chiara sovra performance dei portafogli Small e con alti valori dell'indice Book/Market. 

Un recente studio svolto da Fama e French che considera i rendimenti dal 1991 al 2019 conferma i risultati emersi nell'analisi sopra, mostrando che i rendimenti dei portafogli Big-Low Book/Market e Small-High Book/Market hanno subito un calo rilevante a partire dal 2005. 

Diverse e molteplici sono le interpretazioni fornite a riguardo, per esempio si ipotizza l'esistenza di fattori di rischio che ancora non sono stati scoperti o non sono stati presi in considerazione, oppure si ritiene che i prezzi dei titoli siano mispriced a causa di errori comportamentali. 

Infine alcuni economisti ipotizzano che lo sfruttamento delle anomalie consentirebbe agli investitori di trarre profitti solo successivamente alla loro scoperta, l'efficienza dei mercati infatti condurrebbe nel medio-lungo periodo ad un aggiustamento dei prezzi che precluderebbe agli investitori la possibilità di continuare a ottenere guadagni elevati.  

Ciò spiegherebbe la scomparsa del Size effect e del Value effect nell'ultimo ventennio. 

 