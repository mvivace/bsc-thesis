
\chapter{L'evoluzione dei modelli di pricing}
\section{La moderna teoria del portafoglio}

\subsection{Introduzione}

Uno dei contributi più rilevanti che segna la nascita e lo sviluppo dei modelli di asset allocation e di pricing viene fornito da Harry Markowitz, economista statunitense di fama internazionale. 


Grazie ai suoi elaborati, nel 1990 condivise con Merton Miller e William Sharpe il premio Nobel per l’economia  “Per i contributi pionieristici forniti nell'ambito dell'economia finanziaria”\cite{noauthor_sveriges_nodate}.
	
Gli scritti \textit{Portfolio selection} (1952) \cite{markowitz_portfolio_1978}  e \textit{Portfolio selection: efficient diversification of investments} (1959) \cite{noauthor_portfolio_nodate}  segnano la nascita della Moderna Teoria del Portafoglio.  

Il \textit{portafoglio} è l'insieme di diverse attività $i$ detenute da un investitore \cite{noauthor_portafoglio_nodate}. La quota del portafoglio $w_i$ investita in ciascuna attività è nota come peso di portafoglio e la somma dei pesi soddisfa la seguente condizione:
\begin{equation}
\label{somma pesi}
 \displaystyle\sum_{i=1}^k w_i=1
\end{equation}

L'elaborazione di una nuova strategia attraverso l'introduzione di due nuovi concetti come la \textit{diversificazione} e il criterio\textit{ Media-Varianza} comporta il superamento del precedente approccio basato sulla sola massimizzazione dei rendimenti.

Il processo di diversificazione è una tecnica di costruzione di un portafoglio che consiste nel distribuire il proprio capitale di investimento in una serie di attività per limitare le perdite nel caso uno o più strumenti non abbiano una buona performance. Il criterio Media-Varianza permette di individuare portafogli caratterizzati da massimo rendimento atteso per ogni livello di rischio.

Tali grandezze vengono quantificate attraverso l'utilizzo di due indici statistici: media e varianza. 

Il processo che descrive l'utilizzo e la messa in pratica di questi strumenti è definito portfolio selection.

Gli scritti presentano prima un quadro teorico introducendo i principi generali e le tecniche di analisi del portafoglio, proseguono con la descrizione degli input ed output del modello mostrando i benefici derivanti da una corretta implementazione delle strategie menzionate e dalla diversificazione del portafoglio e mostrano strumenti analitici per determinare la \textit{frontiera efficiente}, l'insieme di tutti i portafogli che massimizzano i rendimenti a parità di varianza. Infine sulla base delle preferenze dell'investitore, date dalla \textit{funzione di utilità}, individuano sulla frontiera efficiente il portafoglio che massimizza l'utilità dell'investitore. 


\subsection{Ipotesi}

Le scelte di investimento di un operatore economico sono soggette ad un particolare grado di incertezza dovuta a fattori micro e macro economici e a condizioni non economiche che spesso influenzano l'andamento dei titoli in modo non banale. 

Per rappresentare il comportamento dei rendimenti vengono quindi utilizzate le \textit{variabili aleatorie} \cite{noauthor_distribuzione_nodate}, le quali assumono valori diversi in corrispondenza di eventi casuali diversi. Data una variabile aleatoria $X$ la sua \textit{distribuzione di probabilità} è una funzione che a un insieme di valori possibili di $X$ associa la rispettiva probabilità secondo una regola precisa. 

Dal momento che i rendimenti sono caratterizzabili utilizzando media e  varianza, si assume che la \textit{distribuzione Gaussiana} che si avvale di questi indici, possa descriverne il comportamento. Tale distribuzione, chiamata anche distribuzione normale viene utilizzata per descrivere variabili aleatorie a valori reali che tendono a concentrarsi intorno al proprio valore medio $\mu$. \`E caratterizzata da un grafico simmetrico rispetto al proprio valore medio e a “campana”.
\begin{figure} [h!]
	\centering
	\includegraphics[width=0.8\linewidth]{"imgs/normale"}
	\caption{Distribuzione normale \cite{noauthor_normal_nodate}}
	\label{fig:normale}
\end{figure}

La distribuzione Gaussiana è altresì unimodale pertanto media, moda e mediana coincidono e al tendere di x verso $+\infty$ o $-\infty$ tende a 0. 

Dalla \ref{fig:normale} possiamo notare che le variazioni della forma, a parità di valore medio, dipendono da una variazione della deviazione standard $\sigma$, mentre gli spostamenti della distribuzione lungo l'asse $x$ dipendono da variazioni del valore medio. 

La funzione di probabilità è la seguente: 
\begin{equation}
f(x)= \frac{1}{\sigma\sqrt{2\pi}}e^{-\dfrac{1}{2}\bigg(\dfrac{x-\mu}{\sigma}\bigg)^2}
\end{equation}
La probabilità che una variabile assuma un valore interno a un intervallo $[a,b]$ è pari all'area della regione sottesa a tale intervallo e viene calcolata come segue: 
\begin{equation}
\int_{a}^{b}{\frac{1}{\sigma\sqrt{2\pi}}e^{-\dfrac{1}{2}\bigg(\dfrac{x-\mu}{\sigma}\bigg)^2}}
\end{equation}

Si presuppone che gli investitori siano avversi al rischio, essi desiderano sempre massimizzare i propri rendimenti e minimizzare il rischio. 

La ragione del superamento del precedente criterio valutativo può essere spiegata attraverso un esempio: ipotizziamo di avere a disposizione ${N}$ titoli $i=1,2,.....,N$ che forniscono un rendimento $R_i$ e che $w_{i}$ siano i pesi del portafoglio.
Per poter massimizzare R sarà necessario scegliere il solo titolo o un portafoglio di titoli che fornisce il rendimento maggiore. Il rendimento del portafoglio è una media dei rendimenti dei titoli ponderata per i pesi del portafoglio: 
\begin{equation}
\label{key}
R_P= \displaystyle\sum_{i=1}^KR_i w_i
\end{equation}
e soggetto a tale condizione: 
\begin{equation}
\label{somma pesi pari a 1}
\displaystyle\sum_{i=1}^K w_i=1
\end{equation}
Se il titolo 1 è quello che presenta maggiore rendimento atteso, il soggetto allocherà tutte le proprie risorse in quel titolo.
Se vi sono molti titoli che forniscono il rendimento massimo qualunque combinazione di questi titoli che rispetti la condizione \ref{somma pesi pari a 1} massimizzerebbe il rendimento R. 

Ipotizziamo per esempio, che un soggetto disponga di 10 titoli $i=1,2....10$ e che il titolo $1$ fornisca il rendimento massimo $R_1=15$. Un soggetto massimizzerebbe il rendimento investendo tutto il capitale nel titolo $1$, l'intero portafoglio sarà pertanto costituito dal titolo $1$:
\begin{equation}
w_i=0 \hspace{0,5cm} \forall i\neq 1 
\end{equation}
\begin{equation}
\label{nondiversificato}
R_P=15\cdot1=15
\end{equation}

Qualora ipotizzassimo invece che 5 di questi titoli forniscano un rendimento pari a $15$ potremmo combinare tali titoli in modo tale che venga soddisfatta la \ref{somma pesi pari a 1}.

\begin{equation}
w_i=0,2 \hspace{0,5cm} i={1,2,3,4,5} 
\end{equation}
Il rendimento del portafoglio $R_P$ sarebbe pari a: 
\begin{equation}
\label{diversificato}
R_P=0,20\cdot15+0,20\cdot 15+0,20\cdot15+0,20\cdot15+0,20\cdot15 = 15
\end{equation}
Si può notare che il rendimento fornito allocando l'intero capitale in un titolo o in un portafoglio di titoli è il medesimo $R_1=R_P=15$, pertanto non è possibile individuare un portafoglio diversificato \ref{diversificato} che sia migliore di un altro non diversificato \ref{nondiversificato}. Si deduce che in precedenza non vi erano strumenti per poter discriminare le diverse soluzioni di investimento.

Inoltre, la costruzione di portafogli costituiti da titoli che forniscono il massimo rendimento non implica necessariamente minima varianza, poiché alcuni titoli possono essere anche ampiamente intercorrelati, prendere in esame un elemento come questo indice può dare beneficio alla diversificazione.  

Le ulteriori assunzioni che vengono fatte per semplificare la modellazione del problema sono le seguenti:
\begin{itemize}
	\item L'orizzonte periodale fa riferimento a un singolo periodo $[t,t+1]$, per esempio pari a 1 anno;
	\item I costi di transazione sono nulli;
	\item Il mercato è perfettamente concorrenziale.
\end{itemize}

\subsection{Input} Illustriamo alcuni elementi fondamentali che saranno utili per giungere alla definizione del portafoglio efficiente.
\paragraph{Rendimento di un titolo}

Il rendimento di un titolo $i$ in un intervallo di un anno $(t, t+1)$ può essere espresso attraverso la seguente formula: 
\begin{equation}\label{key}
{R} = \frac{P_{t+1} -P_{t} + D_{t+1}}{P_{t}}
\end{equation}
Dove $P_{t}$ e $P_{t+1}$ sono rispettivamente i prezzi di chiusura riferiti al periodo $t$ e $t+1$ e $D_\mathrm{t+1}$ è il dividendo relativo al periodo $t+1$.

Il valore ottenuto rappresenta l'ammontare che un investitore otterrebbe o perderebbe se investisse una quota di capitale al tempo $t$, incassasse i dividendi al tempo $t+1$ e rivendesse il titolo al prezzo di chiusura al tempo $t+1$. Se il valore è positivo si parla di profitto, in caso contrario di perdita. 

Questa misura può essere calcolata \textit{ex ante} oppure\textit{ ex post}: nel primo caso si parla di rendimento atteso in quanto viene determinato all'inizio del periodo di investimento, quindi il prezzo di chiusura e il dividendo del periodo $t+1$ sono  ipotizzati al tempo $t$, nel secondo caso si parla di rendimento effettivo, viene calcolato alla fine del periodo di investimento e rappresenta il risultato ottenuto, questa misura può essere utile per confrontare i risultati attesi con quelli effettivi.  

Il rischio deriva da una discrepanza fra il rendimento ex ante e il rendimento ex post in quanto il prezzo e il dividendo del periodo $t+1$ sono due valori ipotizzati nel periodo $t$ e pertanto soggetti a variazione.

\paragraph{Rendimento atteso o valor medio di un titolo }
Ipotizzando che la variabile $R_{i}$ di un titolo ${i}$ assuma un numero $k$ di valori $r_{i,1}$ $r_{i,2}$, ... , $r_{i,k}$ in corrispondenza di $k$ eventi e che le probabilità che tali valori si verifichino siano $p_{i,1}$, $p_{i,2}$, ... , $p_{i,k}$ è possibile calcolare il rendimento atteso del titolo $i$ come media ponderata dei $k$ valori, i cui pesi sono rappresentati dalle rispettive probabilità: 
\begin{equation}
\label{key}
{E(R_{i})}= \displaystyle\sum_{j=1}^k r_{i,j} p_{i,j}
\end{equation}
\paragraph{Rendimento atteso di un portafoglio}
Il rendimento atteso di un portafoglio $E(R_P)$ è la media ponderata dei rendimenti attesi dei titoli $E(R_{i})$, i cui pesi sono rappresentati dalle quote del portafoglio $w_i$ investite in ogni titolo $i$:   
\begin{equation}
\label{key}
{E(R_P)}= \displaystyle\sum_{i=1}^k E(R_\mathrm{i}) w_\mathrm{i}
\end{equation}
Il rendimento atteso del portafoglio può essere espresso anche come combinazione lineare dei rendimenti attesi dei titoli.

\paragraph{Varianza dei titoli}
La varianza di un titolo $i$ misura la dispersione dei valori 

$r_{i,1},r_{i,2},\dots, r_{i,k}$ intorno al proprio valore medio $E(R_{i})$, ponderata per la probabilità che ogni evento si verifichi $p_{i,1}$, $p_{i,2}$, ... , $p_{i,k}$. Tanto più i valori si discostano dal proprio rendimento atteso tanto più l'investitore dovrà sostenere un rischio maggiore. La formula è la seguente:
\begin{equation}
\label{key}
\sigma^{2}_i = \displaystyle\sum_{j=1}^k p_{ij} [r_{ij}-E(R_{i})]^2
\end{equation}
\paragraph{Varianza di un portafoglio} A differenza del rendimento atteso, la media ponderata delle varianze dei titoli contenuti all'interno del portafoglio non rappresenta un'appropriata misura del rischio. \`E infatti necessario catturare anche la relazione che intercorre tra i titoli, la \textit{covarianza} $\sigma_{ij}$. Quest'ultima è una misura della dispersione congiunta dei rendimenti di due titoli $i=1,2$ intorno alla propria media. La formula è la seguente:
\begin{equation}
\label{key}
\sigma_{1,2}= E[(r_{1}-E(R_{1}))(r_{2}- E(R_{2}))]
\end{equation}
La varianza di un portafoglio costituito dai due titoli i cui pesi sono pari a $w_1$ e $w_2$ sarà pertanto: 
\begin{equation}
\label{key}
\sigma^{2}_P =w_{1}^2\sigma^{2}_1+w_{2}^2\sigma^{2}_2 + 2 w_{1}  w_{2}\sigma_{1,2}
\end{equation}
Dalla varianza deriva la deviazione standard che è: 
\begin{equation}
\sigma_P=\sqrt{\sigma^{2}_P}
\end{equation}

Lo studio del segno della covarianza è estremamente importante nella nostra analisi poiché ci consente di valutare se tra le coppie di titoli esiste una relazione positiva negativa o nulla.

\begin{figure}[h!]
	\centering
	\includegraphics[width=1\linewidth]{"imgs/Covariance"}
	\caption{Covarianza \cite{noauthor_variance_nodate}}
	\label{fig:covariance}
\end{figure}


Attraverso i grafici nella figura \ref{fig:covariance} possiamo osservare i diversi scenari, il primo e il secondo grafico (Cov=-5.4 e Cov=-3) mostrano una relazione negativa, i rendimenti si muovono in direzione opposta, il terzo grafico (Cov=0) mostra una relazione nulla, i rendimenti si muovono in maniera indipendente infine il quarto e quinto grafico (Cov= 2 e Cov=4.5), mostrano una relazione positiva, i rendimenti dei due titoli si muovono insieme.  

La covarianza dipende strettamente dalle unità di misura delle due variabili e qualora siano diverse risulta difficile effettuare dei confronti. Per tale ragione in statistica si preferisce utilizzare l'\textit{indice di correlazione} di Pearson che è dato da:
\begin{equation}
\rho_{ij}=\dfrac{\sigma_{ij}}{\sigma_{i}\sigma_{j}}
\end{equation}
L'indice è adimensionale e può assumere valori compresi fra $-1$ e $+1$. 

Come la covarianza, anche la correlazione indica il grado in cui le variabili variano insieme.
\begin{figure}[h!]
	\centering
	\includegraphics[width=0.7\linewidth]{"imgs/correlation 1 (2)"}
	\caption{Correlazione \cite{noauthor_scatter_nodate}}
	\label{fig:correlation-1-2}
\end{figure}


Valori positivi dell'indice indicano l'esistenza di una correlazione lineare positiva, valori negativi dell'indice indicano l'esistenza di una correlazione lineare negativa e il valore pari a zero indica assenza di correlazione.
 
Utilizzando l'indice di correlazione è possibile scrivere la varianza del portafoglio contenente due titoli $A$ e $B$ come segue: 
\begin{equation}
\label{key}
\sigma^{2}_{AB} = w_{A}^2\sigma^{2}_A+w_{B}^2\sigma^{2}_B + 2 w_{A} w_{B} \sigma_A \sigma_B \rho_{A,B}
\nonumber
\end{equation}

\subsection{Benefici derivanti dalla diversificazione}

L'investitore, studiando le correlazioni tra i titoli, può costruire portafogli che offrono un rendimento superiore o un rischio inferiore rispetto ai portafogli costruiti ignorando gli effetti della correlazione. 

Per dimostrare questa affermazione ci avvarremo del rendimento atteso medio ${E(R_{w})}$ e della varianza media $\sigma^{2}_w$ degli $N$ titoli  contenuti all'interno di un portafoglio $P$:  
\begin{equation}
\label{key}
{E(R_{w})}=\dfrac{1}{N} E(R_{p})
\end{equation}
\begin{equation}
\label{key}
\sigma^{2}_w= \dfrac{1}{N} \sigma^{2}_p 
\end{equation}

\paragraph{Variabili aleatorie incorrelate} Ipotizziamo che le variabili siano caratterizzate da: 
\begin{equation}
E= E(r_{1})= E(r_{2})= .......= E(r_{N})
\end{equation}
\begin{equation}
\sigma^{2}= \sigma^{2}_1=\sigma^{2}_2 =.......= \sigma^{2}_N 
\end{equation}
con:
\begin{equation}
\sigma_{r_1, r_2}=0, \sigma_{r_1, r_3}=0,..........,\sigma_{r_1, r_N}=0.
\end{equation}
Siano $s$ e $w$ la somma e la media delle $N$ variabili aleatorie, in tal caso:
\begin{equation}
\begin{split}
s= r_{1}+ r_{2}+....+r_{N} \hspace{0,5cm}e \hspace{0,5cm} w &=\dfrac{r_{1}+ r_{2}+....+r_{N}}{N} \\
w &=\dfrac{s}{N}
\end{split}
\end{equation}
Il valore atteso di $s$ e di $w$ saranno pari a:
\begin{equation}
\begin{split}
Exp(s) &= E_(r_{1})+E_(r_{2})+ ...... + E_(r_{N})\\
 &= N E \\
 \end{split}
 \end{equation}
 \begin{equation}
 \begin{split}
Exp(w)&= Exp\bigg(\dfrac{s}{N}\bigg)\\
& = \dfrac{1}{N}Exp(s) \hspace{0,5cm} \\
& = E
 \end{split}
\end{equation}
La varianza di  $s$ e di $w$ sono date da:
\begin{equation}
\begin{split}
\sigma^{2}(s) & = \sigma^{2}_1+\sigma^{2}_2 +.......+ \sigma^{2}_N\\
& = N \sigma^{2}\\
\end{split}
\end{equation}

\begin{equation}
\begin{split}
\sigma^{2}(w) & =\sigma^{2}\bigg(\frac{s}{N}\bigg) \\
& =\bigg(\frac{1}{N}\bigg)^2 \sigma^{2}(s)\\
& =\frac{1}{N^2} \sigma^{2}(s)
\end{split}
\end{equation}
Ma sappiamo che $\sigma^{2}(s) = N \sigma^{2}$ quindi:
\begin{equation}
\sigma^{2}(w)=\frac{\sigma^{2}(s)}{N} 
\nonumber
\end{equation}
La regola che può essere ricavata da questa dimostrazione è che all'aumentare del numero di titoli inclusi in un portafoglio: 
\begin{itemize}
	\item Il rendimento atteso medio rimane costante;
	\item La varianza media del portafoglio tende a 0.
\end{itemize} 
L'esempio semplifica molto la realtà per diversi motivi:
\begin{itemize}
	\item Abbiamo ipotizzato che i rendimenti dei titoli siano uguali fra loro, nel caso contrario il rendimento atteso medio non sarebbe costante ma sarebbe una media dei rendimenti attesi dei titoli;
	\item Abbiamo ipotizzato che le varianze dei titoli siano uguali fra loro. Diversamente potrebbero presentarsi due casi:  
\begin{description}
	\item[CASO I] La varianza del titolo è inferiore a un valore limite $\sigma*^{2}$, l'investitore riesce a beneficiare della diversificazione.
	\item[CASO II] La varianza dei titoli è superiore al valore limite $\sigma*^{2}$, l'investitore ottiene minori benefici dalla diversificazione. 
\end{description}
	\paragraph{CASO I}

	\begin{equation}
	\begin{split}
	\sigma^{2}(s)= \sigma^{2}_1+\sigma^{2}_2 +.......+ \sigma^{2}_N &\leq
	\sigma*^{2}= \sigma*^{2}+\sigma*^{2}+.......+ \sigma*^{2} \\
	\sigma^{2}(s)&\leq N\sigma*^{2}
	\end{split}	
	\end{equation}

	dal momento che $\sigma^{2}(w)= \frac{\sigma^{2}(s)}{N^2} $ e $\sigma*^{2}(w)= \frac{\sigma*^{2}(s)}{N^2}$

	dalle ipotesi ricaviamo che: 
	\begin{equation}
	\sigma^{2}(w)= \frac{\sigma^{2}(s)}{N^2} \leq \sigma*^{2}(w)= \frac{1}{N^2}\sigma*^{2} N =\frac{\sigma*^{2}}{N}
	\end{equation}
	
	Dato che all'aumentare di N, $\dfrac{\sigma*^{2}}{N}$ tende a 0, allora anche $\sigma^{2}(w)$ tenderà a 0.
	
\paragraph{CASO II}
	
Se l'investitore sceglie di introdurre titoli aggiuntivi con varianza crescente $\sigma^{2}_1\leq\sigma^{2}_2 \leq.......\leq \sigma^{2}_N$ è possibile che gli effetti della diversificazione si riducano. 
	
Ipotizziamo che le varianze possano essere scritte in funzione della varianza del titolo 1: $\sigma^{2}_2=2\sigma^{2}_1, \sigma^{2}_3=3\sigma^{2}_1$,......,$\sigma^{2}_N = N\sigma^{2}_1$, in questo modo è possibile scrivere $\sigma(s)^{2}$ e $\sigma^{2}(w)$:
\begin{equation}
\begin{split}
\sigma^{2}(s) &= \sigma^{2}_{1}+2\sigma^{2}_{1}+3\sigma^{2}_{1}+....+N\sigma^{2}_{1}\\
&= \sigma^{2}_{1}(1+2+3+.....+N)
\end{split}
\end{equation}
La somma di N numeri può essere scritta come: 
\begin{equation}
1+2+3+....+N= \dfrac{N(N+1)}{2}
\nonumber
\end{equation}
\begin{equation}
\begin{split}
\sigma^{2}(s) & =\sigma^{2}_{1}\dfrac{N(N+1)}{2}
\nonumber
\end{split}
\end{equation}
\begin{equation}
\begin{split}
\sigma^{2}(w) & = \frac{1}{N^2}\sigma^{2}(s)\\
& =\frac{1}{N^2}\dfrac{N(N+1)}{2}\sigma^{2}_{1}\\
& = \frac{N+1}{N}\frac{\sigma^{2}_{1}}{2}
\end{split}
\end{equation}

All'aumentare di N, $\dfrac{N+1}{N}$ tende a 1 e $\sigma^{2}(w)$ tende a $\dfrac{\sigma^{2}_{1}}{2}$. 

Qualora le varianze dei titoli siano inferiori ad un certo margine e i titoli siano incorrelati fra loro il processo di diversificazione consente di ridurre la varianza media a 0 (CASO I), viceversa gli effetti benefici della diversificazione sono di minore entità $\frac{\sigma^{2}_{1}}{2}$(CASO II).
	\item Abbiamo considerato assenza di correlazione, nella realtà i rendimenti dei titoli sono correlati.
\end{itemize}

\paragraph{Variabili aleatorie correlate}

Il numero di covarianze associate a N variabili, escludendo le varianze e sapendo che   
$\sigma_{i,j}$=$\sigma_{j,i}$, è pari a:
\begin{equation}
1+2+3+....+(N-1)= \dfrac{N(N-1)}{2}
\nonumber
\end{equation}
Per esempio se un portafoglio è costituito da 3 titoli, il numero di covarianze è pari a 1+2=3 e sono $\sigma_{1,2}$, $\sigma_{1,3}$ e $\sigma_{2,3}$. La covarianza media è data da:
\begin{equation}
\begin{split}
covarianza\hspace{0,1cm}media & = \frac{somma \hspace{0,1cm} delle \hspace{0,1cm} covarianze}{numero\hspace{0,1cm} delle \hspace{0,1cm}covarianze} \\
&=\dfrac{somma\hspace{0,1cm} delle \hspace{0,1cm}covarianze}{\frac{N(N-1)}{2}}\\
&=\dfrac{somma\hspace{0,1cm} delle\hspace{0,1cm} covarianze\hspace{0,1cm}  2}{N(N-1)}
\end{split}
\end{equation}
Viceversa la somma delle covarianze può essere espressa come: 
\begin{equation}
somma\hspace{0,1cm} delle\hspace{0,1cm} covarianze\hspace{0,1cm} = covarianza\hspace{0,1cm} media\hspace{0,1cm} \frac{N(N-1)}{2}
\end{equation}
La varianza della somma di variabili con covarianza $\neq 0$ è data da:

\begin{equation}
\begin{split}
\sigma^{2}(s) & = \sigma^{2}_1+\sigma^{2}_2 +.......+ \sigma^{2}_N \\
&+2\sigma_{1,2} +2\sigma_{1,3} +....+ 2\sigma_{1,N} + \\
&+2\sigma_{2,3} +2\sigma_{2,4} +....+ 2\sigma_{2,N} +....+ etc
\end{split}
\end{equation}
Questo potrebbe essere riassunto nel seguente modo: 
\begin{equation}
\begin{split}
\sigma^{2}(s) & = somma\hspace{0,1cm} delle\hspace{0,1cm} varianze + 2 somma\hspace{0,1cm} delle\hspace{0,1cm} covarianze \\
\sigma^{2}(w) &= \frac{\sigma^{2}(s)}{N^2}\\
&= \frac{somma\hspace{0,1cm} delle\hspace{0,1cm} varianze}{N^2}  + 2\frac{somma\hspace{0,1cm} delle\hspace{0,1cm} covarianze }{N^2} \\
& = \frac{somma\hspace{0,1cm} delle\hspace{0,1cm}varianze}{N^2}
+ \hspace{0,1cm} media\hspace{0,1cm} delle\hspace{0,1cm} covarianze\hspace{0,1cm} {\dfrac{N(N-1)}{2}}
\end{split}
\end{equation}
Ipotizzando che la varianza dei titoli sia inferiore ad un certo margine (CASO I) il primo membro della sommatoria è esattamente pari a $\sigma^{2}(w)$ quindi tende a 0, mentre ${\frac{N(N-1)}{2}}$ tende ad 1, ciò significa che $\sigma^{2}(w)$ tenderà ad un valore prossimo alla media delle covarianze. 

Riassumendo, la diversificazione consente all'investitore di azzerare i rischi nel caso in cui il portafoglio contenga titoli non correlati, viceversa riduce solo parzialmente il rischio del portafoglio.

Per capire il peso della covarianza nelle scelte di investimento possiamo riportare il seguente esempio: supponiamo di dover scegliere se inserire il titolo A o il titolo B in un portafoglio di 99 titoli. Il titolo A ha una varianza pari a $\sigma^{2}$ e una covarianza con gli altri 99 titoli pari a $0,5\sigma^{2}$ mentre il titolo B ha una varianza pari a $25\sigma^{2}$ mentre il suo rendimento è incorrelato ai rendimenti degli altri titoli. 

Sapendo che l'influenza di un singolo titolo sulla varianza del portafoglio è data da: 
\begin{equation}
\dfrac{1}{N^2}[varianza \hspace{0,1cm}del\hspace{0,1cm} portafoglio + 2(somma \hspace{0,1cm}delle \hspace{0,1cm}covarianze)]
\end{equation}
\begin{equation}
Influenza_A= \frac{\sigma^{2}+2\cdot99\cdot0,5\sigma^{2}}{10.000}=0,01\sigma^{2}
\end{equation}
\begin{equation}
Influenza_B= \frac{25\sigma^{2}+2\cdot99\cdot0\sigma^{2}}{10.000}=0,0025\sigma^{2}
\end{equation}
Come si può intuire l'aggiunta all'interno del portafoglio di un titolo rischioso come il titolo $B$ non implica un aumento della varianza del portafoglio e ciò è dovuto all'influenza della covarianza.

Dunque il processo di selezione del portafoglio diviene un problema di minimizzazione della covarianza fra i titoli.

\subsection{Criterio media-varianza e frontiera efficiente}
Attraverso il \textit{criterio media-varianza}, che si avvale di tutti i concetti illustrati, possiamo costruire la \textit{frontiera efficiente}. 

Dato un set di portafogli, un portafoglio è definito inefficiente quando è possibile ottenerne un altro che garantisce un rendimento maggiore a parità di rischio o comporta un rischio inferiore a parità di rendimento atteso. 

Per semplificare l'analisi vedremo come individuare la frontiera efficiente considerando un portafoglio con 3 titoli in cui $X_{1}$, $X_{2}$ e $X_{3}$ rappresentano la proporzione del portafoglio investita rispettivamente nel titolo $1,2,3$ in modo tale che soddisfino le seguenti condizioni: 
\begin{equation}
\label{Condizioneparia1}
X_{1}+X_{2} + X_{3}=1
\end{equation}
\begin{equation}
X_{1} \geq  0
\nonumber
\end{equation}
\begin{equation}
X_{2}\geq  0
\nonumber
\end{equation}
\begin{equation}
X_{3}\geq  0
\end{equation}
$X_{3}$ può essere scritto come:
\begin{equation}
X_{3}= 1- X_{1}- X_{2}
\end{equation}

Tutte le combinazioni che possono essere costruite con i 3 titoli sono incluse nel triangolo $abc$ della \ref{fig:geometric-rapresentation-of-portfolios} e sono definite\textit{ legittime} perché soddisfano la \ref{Condizioneparia1}.

\begin{figure}[h]
	\centering
	\includegraphics[width=0.4\linewidth]{"imgs/Geometric rapresentation of portfolios"}
	\caption{Rappresentazione geometrica dei portafogli legittimi}
	\label{fig:geometric-rapresentation-of-portfolios}
\end{figure}

\newpage

\paragraph{Iso-mean lines}
Siano $\mu_1$, $\mu_2$, $\mu_3$ i rendimenti attesi dei titoli è possibile calcolare il rendimento atteso del portafoglio come segue:
\begin{equation}
\label{key}
E= \displaystyle\sum_{i=1}^3 \mu_{i} X_{i}
\end{equation}
\begin{equation}
\label{key}
E= \mu_{1} X_{1}+\mu_{2} X_{2}+\mu_{3} X_{3}
\end{equation}
Se sostituiamo $X_{3}= 1- X_{1}- X_{2}$ possiamo riscrivere il rendimento atteso in funzione di 2 variabili: 
\begin{equation}
\label{key}
E= \mu_{1} X_{1}+\mu_{2} X_{2}+\mu_{3} (1- X_{1}- X_{2})
\nonumber
\end{equation}
\begin{equation}
\label{key}
E= X_{1}   (\mu_{1}-\mu_{3})+X_{2}   (\mu_{2}-\mu_{3})+\mu_{3}
\nonumber
\end{equation}
Ipotizzando per esempio: $\mu_{1}=0,10, \mu_{2}=0,05, \mu_{3}= 0,07$ 
\begin{equation}
\label{key}
E= 0,03    X_{1} -0,02    X_{2}+0,07
\nonumber
\end{equation}
Qualunque portafoglio con rendimento atteso pari a 0,08 soddisferà la seguente equazione: 
\begin{equation}
\begin{split}
0,08&= 0,03  X_{1} -0,02  X_{2}+0,07\\
0,01 &= 0,03  X_{1} -0,02  X_{2}
\nonumber
\end{split}
\end{equation}

Questa equazione è rappresentata graficamente da una \textit{ISO-linea}. Una iso-linea è il luogo dei punti (i portafogli) che presentano stesso rendimento atteso. 

\begin{figure}[h!]
	\centering
	\includegraphics[width=0.4\linewidth]{"imgs/isomean line"}
	\caption{Iso-mean lines}
	\label{fig:isomean-line}
\end{figure}

Il vettore in figura \ref{fig:isomean-line} indica la direzione verso la quale si muovono le ISO-linee. Se ci muoviamo nella direzione indicata troviamo ISO-linee a cui sono associati rendimenti crescenti. 

In questo caso specifico in cui $\mu_{1}\neq\mu_{2}\neq\mu_{3}$, le ISO-linee formano un sistema di rette parallele. 

Se invece $\mu_{1}=\mu_{2}=\mu_{3}$ tutti i portafogli avrebbero il medesimo rendimento atteso e l'unico portafoglio ottimale sarebbe quello con minima varianza. 

\paragraph{Iso-curve lines}

La varianza di un portafoglio costituito da 3 titoli può essere espressa come segue:
\begin{equation}
\label{key}
\sigma^{2}_{p} = X_{1}^2 \sigma^{2}_{1}+X_{2}^2 \sigma^{2}_{2}+X_{3}^2 \sigma^{2}_{3} + 2  X_{1}  X_{2}\sigma_{1,2}+2  X_{1}  X_{3}  \sigma_{1,3}+2  X_{2}  X_{3}  \sigma_{2,3} 
\end{equation}
Se sostituiamo $X_{3}= 1- X_{1}- X_{2}$ possiamo riscrivere la varianza in funzione di 2 variabili: 
\begin{equation}
\begin{split}
\label{key}
\sigma^{2}_{p} & =X_{1}^2(\sigma^{2}_{1}-2\sigma_{1,3}+\sigma^{2}_{3})+X_{2}^2(\sigma^{2}_{2}-2\sigma_{2,3}+\sigma^{2}_{3})\\
&+2X_{1}X_{2}(\sigma_{1,2}-\sigma_{1,3}-\sigma_{2,3}+\sigma^{2}_{3}) \\
&+2X_{1}(\sigma_{1,3}-\sigma^{2}_{3})+2X_{2}(\sigma_{2,3}-\sigma^{2}_{3})+\sigma^{2}_{3}
\nonumber
\end{split}
\end{equation}
Per i tre titoli con:
\begin{equation}
\begin{split}
\sigma^{2}_{1}&=\sigma^{2}_{2}=0,01 \\
\sigma^{2}_{3}&=0,04\\
\sigma_{1,2}&=0,05\\
\sigma_{1,3}&=\sigma_{2,3}=0\\
\nonumber
\end{split}
\end{equation}
La varianza sarà pari a:
\begin{equation}
\sigma^{2}_{p}= 0,05X_{1}^2 + 0,05X_{2}+0,09X_{1}X_{2}-0,08X_{1}-0,08X_{2} +0,04
\nonumber
\end{equation}
Tutti i portafogli con varianza pari a 0,01 soddisfaranno la seguente equazione:
\begin{equation}
0,01= 0,05X_{1}^2 + 0,05X_{2}+0,09X_{1}X_{2}-0,08X_{1}-0,08X_{2} +0,04
\nonumber
\end{equation}
Questa equazione è rappresentata graficamente da una \textit{ISO-curva}. Una ISO-curva è il luogo dei punti ossia dei portafogli che presentano la stessa varianza. In questo caso l'iso-curva è rappresentata da un'ellisse. 

Tutte le ellissi sono caratterizzate da stesso centro, stessa direzione e stesso rapporto fra il diametro maggiore e il diametro minore. 
\begin{figure}[h!]
	\centering
	\includegraphics[width=0.4\linewidth]{"imgs/iso variance curve"}
	\caption{Iso-variance curves}
	\label{fig:iso-variance-curve}
\end{figure}

\newpage
Il punto c nella \ref{fig:iso-variance-curve} rappresenta il portafoglio con varianza minore.

Mettendo insieme i due grafici trovati è possibile individuare la cosiddetta critical line $ll$, la linea congiunge tutti portafogli che per ogni livello di rendimento presentano minima varianza:

\begin{figure}[h!]
	\centering
	\includegraphics[width=0.4\linewidth]{"imgs/critical line"}
	\caption{La critical line}
	\label{fig:critical-line}
\end{figure}

Se un punto giace sulla critical line allora minimizza la varianza, viceversa se un punto minimizza la varianza allora giace sulla critical line. 

Tuttavia, non tutti i punti sulla critical line sono efficienti e legittimi. Per esempio il punto $d$ che ha un rendimento $E_2$ ed una varianza $V_1$ è inefficiente in quanto è possibile trovare un altro punto, per esempio $e$ che a parità di rischio presenta rendimento maggiore, per tale ragione si dice che il portafoglio $d$ è “dominato” dal portafoglio $e$.  Attraverso il criterio Media-Varianza, il quale suggerisce di scegliere tutti i portafogli che a parità di rischio offrono un rendimento maggiore o a parità di rendimento offrono un rischio inferiore, è possibile definire la frontiera di portafogli efficienti o “dominanti”.

\newpage
\paragraph{Frontiera efficiente con 3 titoli} 

La figura \ref{fig:the-set-of-efficient-portfolios} include diverse informazioni che riguardando l'analisi di un portafoglio contenente 3 titoli: 
\begin{figure} [h!]
	\centering
	\includegraphics[width=0.4\linewidth]{"imgs/The set of efficient portfolios"}
	\caption{Il set di portafogli efficienti}
	\label{fig:the-set-of-efficient-portfolios}
\end{figure}


\begin{itemize}
	\item Il punto c rappresenta il portafoglio con minima varianza rispetto a tutti gli altri portafogli legittimi e non legittimi, in questo caso il portafoglio c è legittimo;
	\item Il punto $\bar{X}$ rappresenta il portafoglio che consente di ottenere il massimo rendimento atteso rispetto a qualunque altro portafoglio legittimo;
	\item La direzione delle ISO-line;
	\item La critical line.
\end{itemize}

Il set di portafogli efficienti è dato dalla linea $ca\bar{X}$ che congiunge il portafoglio c con il portafoglio $\bar{X}$. 

Affinché un portafoglio P sia efficiente è necessario che soddisfi le seguenti condizioni: 
\begin{itemize}
	\item P deve essere un portafoglio legittimo;
	\item Qualunque altro portafoglio legittimo che ha un rendimento atteso superiore a P deve avere una varianza superiore a P;
	\item Qualunque altro portafoglio legittimo che presenta varianza inferiore rispetto a P deve necessariamente avere un rendimento atteso inferiore. 
\end{itemize}
Possiamo generalizzare il problema utilizzando tutti i titoli presenti sul mercato, le combinazioni Media-Varianza che l'investitore potrebbe costruire sono rappresentate nel seguente grafico. La frontiera efficiente è data da un arco di circonferenza compreso fra due punti della stessa:
\newpage
\begin{figure} [h!]
	\centering
	\includegraphics[width=0.4\linewidth]{"imgs/efficient combination"}
	\caption{Frontiera efficiente}
	\label{fig:efficient-combination}
\end{figure}
\paragraph{Funzione di utilità}
Gli investitori sono caratterizzati da un diverso grado di avversione al rischio e da diverse preferenze, pertanto la scelta del portafoglio ottimale si baserà sulla funzione di utilità e sulla frontiera efficiente. Markowitz utilizza una funzione di utilità quadratica che dipende dal rendimento atteso $r$ e dal grado di di avversione al rischio $A$. Tanto maggiore è la varianza $r^2$ tanto minore è l'utilità dell'investitore.
\begin{equation}
U= r-Ar^2
\end{equation}

Graficamente le funzioni di utilità sono rappresentate da curve di indifferenza. La curva di indifferenza è l'insieme dei portafogli che forniscono la medesima utilità all'investitore. Tanto più una curva si allontana verso nord-ovest tanto maggiore sarà l'utilità dell'investitore. Il portafoglio ottimale sarà il punto di tangenza fra la curva di indifferenza più alta e la frontiera efficiente. 
\begin{figure} [h!]
	\centering
	\includegraphics[width=0.4\linewidth]{"imgs/indiff curve"}
	\caption{Il portafoglio ottimale \cite{noauthor_optimum_nodate}}
	\label{fig:indiff-curve}
\end{figure}
\subsection{Validità del modello di Markowitz}
Il modello di Markowitz rappresenta senz'altro un enorme evoluzione, tuttavia presenta dei limiti che ne rendono difficile l'applicazione.  Per esempio il numero di dati input necessario per calcolare il rischio di un portafoglio è molto elevato, ipotizzando che sia costituito da 100 titoli è necessario calcolare 100 varianze e 4950 covarianze \cite{kritzman_what_1991}. Inoltre Markowitz si limita a considerare solo titoli rischiosi escludendo una delle possibilità di investimento, il titolo risk-free. 

Quando \textit{Portfolio Selection} (1952) venne pubblicato per la prima volta tali limiti rappresentarono un forte ostacolo per la sua accettazione da parte di altri economisti.

Sono stati proprio questi limiti a incentivare lo sviluppo di un nuovo modello da parte di William Sharpe: il single index model che consente di semplificare le procedure per definire il portafoglio ottimale. 

\section{Il Capital Asset Pricing Model}
\subsection{Introduzione}

Grazie al modello Media-Varianza e al Teorema di Separazione di Tobin  (1958) fu possibile costruire uno dei primi modelli di Asset Pricing \cite{fama_capital_2004}. Tale modello prende il nome di Capital Asset Pricing Theory (CAPM) ed è stato sviluppato inizialmente da William Sharpe (1964), per poi essere rielaborato da John Lintner (1965) e Jan  Mossin (1966).

Il CAPM si propone di spiegare in che modo i rendimenti di un titolo  sono influenzati dal rischio ad esso associato. L'approccio si basa sulla presenza di due tipologie di rischio:

\begin{itemize}
	\item \textit{Rischio sistematico} o \textit{di mercato} associato a condizioni o a variabili economiche che influiscono sull'andamento del mercato come ad esempio inflazione, tassi di interesse, cicli economici, ecc. e che non può essere ridotto tramite la diversificazione;
	\item \textit{Rischio non sistematico} o \textit{di impresa} deriva dalle caratteristiche specifiche dell'impresa e settore di appartenenza che può essere eliminato tramite la diversificazione
\end{itemize} 

Essendo il rischio specifico completamente diversificabile il rendimento di un titolo dovrà essere influenzato dal solo rischio sistematico catturato dal fattore $\beta$. 


\paragraph{Teorema della separazione di Tobin}
Tale teorema consente di superare uno dei limiti del modello di Markowitz, l'assenza di attività prive di rischio. 

Tobin si chiede come verrebbero influenzate le scelte di investimento qualora venisse introdotto un titolo privo di rischio nel portafoglio  \cite{tobin_liquidity_1958}  \cite{damodaran_investment_2002} \cite{ho_oxford_2004} \cite{jain_financial_2007}. Gli investitori totalmente avversi al rischio potrebbero scegliere un portafoglio interamente costituito dal titolo privo di rischio. Viceversa, un investitore meno avverso al rischio potrebbe decidere di includere anche titoli rischiosi. 

L'introduzione del titolo privo di rischio consente di costruire una retta definita \textit{Capital Allocation Line} (CAL) su cui giacciono tutti i portafogli costituiti dal titolo privo di rischio e da uno dei portafogli rischiosi che giacciono sulla frontiera efficiente.

Ipotizzando la possibilità di creare un portafoglio \textit{C} costituito da un titolo privo di rischio $f$ e da un portafoglio rischioso $A$, definiamo $w_f$ la quota del portafoglio investita nel titolo $f$ mentre $(1-w_f)$ quella investita nel portafoglio $A$. Il rendimento e il rischio del portafoglio \textit{C} sono dati da:
\begin{equation}
E(R_C)= w_f E(R_{f})+(1-w_f) E(R_{A})
\end{equation}
\begin{equation}
\sigma_{C}=\sqrt{w_f^2\sigma^2_{f}+(1-w_f)^2\sigma^2_{A}+ 2w_f(1-w_f)\sigma_{f}\sigma_{A}\rho_{f,A}} 
\end{equation}
Dal momento che il rischio legato al titolo risk-free è: $\sigma^2_{f}=0$ \\
ne deriva che:
\begin{equation}
\sigma_{C}=\sqrt{(1-w_f)^2\sigma^2_{A}}=(1-w_f)\sigma_{A}
\nonumber
\end{equation} 
quindi $w_f=1- \frac{\sigma_{C}}{\sigma_{A}}.$\\

Sostituendo $w_f$ all'intero della formula del rendimento atteso abbiamo: 
\begin{equation}
\label{FormularendimentoTobin}
E(R_{C})= E(R_{f})+\frac{E(R_{A})-E(R_{f})}{\sigma_{A}}\sigma_{C}
\end{equation}
Dove: 
\begin{equation}
\label{IndicediSharpe}
S=\frac{E(R_{A})-E(R_{f})}{\sigma_{A}}
\end{equation}
rappresenta l'inclinazione della retta. 

La formula \ref{IndicediSharpe} prende il nome di \textit{premio per il rischio} ed è il rendimento in eccesso che un investitore ottiene per ogni unità di rischio aggiuntiva. Dalla \ref{FormularendimentoTobin} ricaviamo che il rendimento di un portafoglio $C$ è funzione del rendimento del titolo privo di rischio e del premio per il rischio e tale relazione può essere rappresentata graficamente dalla CAL:
\begin{figure} [h!]
	\centering
	\includegraphics[width=0.5\linewidth]{"imgs/Figura+Il+portafoglio+tangente+o+efficiente (2)"}
	\caption{ Capital Allocation Line \cite{berk_finanza_2012}}
	\label{fig:figurailportafogliotangenteoefficiente-2}
\end{figure}

Come possiamo vedere le rette (CAL) possono essere molteplici a seconda del portafoglio rischioso preso in considerazione sulla frontiera efficiente, tuttavia non tutte massimizzano il rendimento atteso dell'investitore. Dunque, il problema che dobbiamo affrontare è l'individuazione del portafoglio rischioso sulla frontiera efficiente che, dato il titolo privo di rischio,  massimizza il rendimento dell'investitore.

Utilizzando la \ref{IndicediSharpe} possiamo calcolare per ogni portafoglio sulla frontiera efficiente il premio per il rischio. Dai calcoli emerge che il portafoglio che consente di massimizzare tale valore è il portafoglio “tangente o efficiente”, pertanto la CAL che prendiamo in considerazione è quindi quella tangente al portafoglio efficiente. 

A questo punto gli investitori sceglieranno uno dei portafogli che giacciono su tale retta in base al loro grado di avversione al rischio. 

Possiamo considerare due casi estremi: un primo caso in cui gli investitori sono totalmente avversi al rischio quindi scelgono il portafoglio costituito interamente dal titolo privo di rischio, tale portafoglio corrisponde al punto $(R_{C})=5\%$, ed un secondo caso in cui gli investitori sono poco avversi al rischio quindi scelgono il portafoglio costituito interamente dal portafoglio tangente. 


Tra questi due estremi vi sono dei casi intermedi in cui gli investitori creano una combinazione costituita sia dal titolo privo di rischio sia dal portafoglio tangente. Questi portafogli si differenziano per la diversa proporzione di titolo risk-free e portafoglio rischioso scelta dall'investitore. 

Infine, a destra del portafoglio tangente si trovano tutti i portafogli costituiti dal portafoglio “efficiente” che vengono acquistati dall'investitore indebitandosi al tasso risk-free. 

Il teorema di separazione afferma che il processo di selezione del portafoglio consta di due fasi distinte: 
\begin{itemize}
	\item Determinazione del portafoglio rischioso ottimale;
	\item Determinazione del mix ideale tra portafoglio rischioso ottimale e titolo privo di rischio.
\end{itemize}
Tale distinzione comporta una totale indipendenza dell'avversione al rischio dell'investitore dalla scelta del portafoglio rischioso efficiente. Il portafoglio efficiente che gli investitori includono all'interno del proprio mix è il medesimo per tutti. 

La combinazione ideale per ogni investitore differisce dalle altre soltanto nella proporzione in cui intendono distribuire il loro capitale tra il titolo privo di rischio e il portafoglio rischioso efficiente. Tale mix dipende strettamente dall'avversione al rischio del soggetto. 

Sotto l'assunzione di equilibrio dei mercati è possibile ipotizzare che il portafoglio desiderato da tutti sia un portafoglio che include tutti i titoli presenti sul mercato nella medesima proporzione in cui tali titoli sono offerti sul mercato. La proporzione è data dal rapporto fra capitalizzazione di mercato (prodotto fra prezzo delle azioni e numero delle azioni in circolazione) e il valore totale di mercato. Tutti i titoli dovranno essere contenuti all'interno di tale portafoglio perché essendoci equilibrio, la domanda deve essere pari all'offerta e l'unico portafoglio che gli investitori domandano è quello offerto sul mercato. Quindi il portafoglio “tangente o efficiente” comune a tutti gli investitori viene definito portafoglio di mercato e la Capital Allocation Line prende il nome di \textit{Capital Market Line} (CML):

\begin{figure}[h!]
	\centering
\includegraphics[width=0.7\linewidth]{"imgs/CML"}
\caption{Capital Market Line \cite{noauthor_kapitalmarktlinie_2019}}
\label{fig:cml}
\end{figure}

\subsubsection{Il Capital Asset Pricing Model}

Per comprendere tutti gli aspetti del CAPM possiamo innanzitutto avvalerci del \textit{Single Index Model}.

\paragraph{Il single-index model}

Gli index models sono modelli progettati per stimare le due componenti del rischio: il rischio di mercato e il rischio specifico. 

Per semplificare l'analisi, si ipotizza che solo il rischio di mercato influenza il rendimento dei titoli e questo fattore può essere rappresentato dal rendimento di un \textit{indice di mercato}.


L'indice di mercato è un paniere di titoli. Il suo valore riflette le variazioni dei prezzi dei titoli contenuti in esso e viene utilizzato per misurare l'andamento di un segmento o di uno specifico mercato. Per esempio ad un miglioramento delle prestazioni delle società quotate corrisponderà anche un miglioramento delle prestazioni dell'indice azionario.
Tra gli indici più noti troviamo il Dow Jones, lo Standard and Poor's 100 e lo Standard and Poor's 500, FTSE MIB, ecc. 

Sia $R_{i}=r_{i}-r_{f}$ il rendimento in eccesso di un titolo $i$ rispetto al titolo privo di rischio, possiamo esprimere tale valore in funzione di 3 fattori: 
\begin{equation}
R_{i}=\beta_{i}R_{M}+e_{i}+ \alpha_{i}
\end{equation}

$R_{M}$ rappresenta il rendimento in eccesso di un indice di mercato rispetto al titolo privo di rischio, una variazione di questo valore riflette un cambiamento nelle variabili economiche oppure eventi macroeconomici che colpiscono il mercato.  

$\beta$ misura il comportamento di un titolo rispetto al mercato. $\beta> 1$ indica che i rendimenti variano in modo più che proporzionale rispetto all'indice di mercato, $\beta<1$ indica che i rendimenti variano in modo meno che proporzionale rispetto all'indice di mercato. 

Il termine $e_{i}$ rappresenta il rischio specifico o diversificabile. 

Infine il termine $\alpha_{i}$ è la componente erratica del rendimento indipendente dal rendimento di mercato. In equilibrio tale fattore è pari a 0. Se $\alpha>0$, il titolo viene definito attraente poiché a parità di rischio offre un rendimento atteso maggiore rispetto ad un altro titolo. Viceversa un $\alpha<0$ indica che, a parità di rischio, il rendimento atteso è inferiore ciò significa che è sovra prezzato. 

Una situazione in cui  $\alpha\neq0$ non persiste per un lungo periodo. La tendenza ad acquistare il titolo sotto prezzato e a vendere quello sovra prezzato condurranno i prezzi dei titoli ai livelli di equilibrio. 

L'index model consente di scomporre il rendimento atteso di un titolo in micro (specifici dell'impresa) e macro (sistematici) elementi. Il rendimento in eccesso è la somma di 3 componenti e dal momento che il rendimento specifico d'impresa è incorrelato con il rendimento di mercato possiamo scrivere la varianza del titolo come segue: 
\begin{equation}
\begin{split}
Variance(R_{i}) &= Variance(\alpha_{i}+\beta_{i}R_{M}+e_{i}) \\
&= Variance(\beta_{i}R_{M})+Variance(e_{i})\\
& = \beta^2_{i}\sigma^2_{M} +\sigma^2(e_{i}) \\
&= Rischio \hspace{0,1cm}  sistematico + Rischio \hspace{0,1cm}  specifico
\end{split}
\end{equation}
\subsection{Ipotesi}

Le assunzioni che consentono al CAPM di garantire mercati dei titoli competitivi dove gli investitori scelgono i medesimi portafogli efficienti attraverso l'utilizzo del criterio media-varianza sono le seguenti:
\begin{itemize}
	\item I mercati dei titoli sono perfettamente competitivi e da essi tutti gli investitori possono trarre profitto in egual misura:
	\begin{itemize}
		\item 	Nessun investitore possiede un capitale sufficientemente alto da influenzare, con le sue azioni, i prezzi dei titoli sul mercato;
		\item Tutte le informazioni relative ai titoli sono disponibili senza sostenere nessun costo;
		\item Non sono previste tasse sui rendimenti e costi di transazione;
		\item Tutti i soggetti possono prestare e prendere a prestito in modo illimitato al tasso risk-free.
	\end{itemize}
\item Gli investitori si differenziano tra loro tranne per il loro capitale iniziale e la loro avversione al rischio:
\begin{itemize}
	\item Tutti gli investitori pianificano i propri investimenti in un medesimo orizzonte temporale;
	\item Gli investitori sono razionali, seguono il criterio media-varianza;
	\item Gli investitori hanno aspettative omogenee in quanto hanno tutti accesso alle medesime informazioni rilevanti, utilizzano gli stessi input e considerano gli stessi set di portafogli. 
\end{itemize}
\end{itemize}

\subsection{Security Market Line}

Una volta individuata la Capital Market Line e ipotizzando che il portafoglio di mercato possa essere rappresentato da un indice di mercato il CAPM definisce i rendimenti dei titoli in funzione di un solo fattore di rischio, quello di mercato. 
Il contributo di un singolo asset $i$ il cui peso è pari a $w_i$ sulla varianza del portafoglio di mercato $M$ è dato da:
\begin{equation}
\sigma^{2}_{M}= w^2_{i}\sigma^{2}_{i}+ (1-w_{i})^2\sigma^{2}_{M}+ 2 w_{i}(1-w_{i})(\sigma_{i,M})
\end{equation}

Essendo il portafoglio costituito da un'ampia gamma di titoli il peso $w_{i}$ tende a 0 viceversa $(1-w_{i})$ tende a 1: 
\begin{equation}
\label{varianza portafoglio di mercato}
\sigma^{2}_{M}= \sigma^{2}_{M}+(1-w_{i}) w_{i}\sigma_{i,M}
\end{equation}
Come possiamo notare dalla \ref{varianza portafoglio di mercato} la varianza del portafoglio di mercato è influenzata dalle sole covarianze dei titoli rispetto al portafoglio di mercato. 

Siano $w_i$ i pesi dei titoli $i=1,2,3,4,...,N$, la varianza di mercato può essere calcolata come segue:
\begin{equation}
\sigma^{2}_{M}= w_{1}\sigma_{1,M}+w_{2}\sigma_{2,M}+w_{3}\sigma_{3,M}+w_{4}\sigma_{4,M} +....+ w_{N}\sigma_{N,M}
\end{equation}
Per avere una misura confrontabile dell'influenza del titolo sulla varianza del portafoglio di mercato possiamo dividere le covarianze per la varianza del portafoglio di mercato.:
\begin{equation}
1=w_{1}\frac{\sigma_{1,M}}{\sigma^{2}_{M}}+w_{2}\frac{\sigma_{2,M}}{\sigma^{2}_{M}}+w_{3}\frac{\sigma_{3,M}}{\sigma^{2}_{M}}+w_{4}\frac{\sigma_{4,M}}{\sigma^{2}_{M}} +....+ w_{N}\frac{\sigma_{N,M}}{\sigma^{2}_{M}}
\end{equation}
dove possiamo scrivere:

\begin{equation}
\beta_i= \frac{\sigma_{i,M}}{\sigma^{2}_{M}} \hspace{0,5cm} \forall i={1,2,3,....,N} 
\end{equation}

tale valore rappresenta la sensitività dei rendimenti di un titolo alle variazioni del portafoglio di mercato. 

L'unico rischio per cui gli investitori richiedono un premio cioè un rendimento aggiuntivo rispetto al risk-free rate è il rischio sistematico, cioè quel rischio che non è possibile limitare tramite la diversificazione.

Perciò il rendimento di ogni titolo contenuto nel portafoglio di mercato è proporzionale al proprio rischio sistematico e al premio per il rischio di mercato. 

Se per esempio un titolo $i$ ha un'influenza due volte superiore rispetto ad un altro titolo $j$ sulla varianza del portafoglio di mercato, dovrà garantire un premio per il rischio due volte superiore rispetto al rendimento del titolo $j$ quindi siano  $\beta_{i}=2$ e  $\beta_{j}=1$ allora dovrà valere:
\begin{equation}
\begin{split}
\dfrac{E(R_{i})-r_{f}}{2} & =\dfrac{E(R_{j})-r_{f}}{1} \\
E(R_{i})-r_{f} & = 2[E(R_{j})-r_{f}]
\end{split}
\end{equation}
In altre parole, ipotizzando un equilibrio di mercato per ogni titolo dovrà valere:
\begin{equation}
\label{uguaglianza sml}
\dfrac{E(R_{i})-r_{f}}{\beta_{i}} =\dfrac{E(R_{M})-r_{f}}{\beta_{M}}
\end{equation}
Dato che la covarianza tra portafoglio di mercato e se stesso è pari alla sua varianza, il beta del portafoglio di mercato assume valore 1.

Possiamo riscrivere la  \ref{uguaglianza sml}:
\begin{equation}
\label{SmL1}
\begin{split}
E(R_{i})-r_{f} & =\beta_{i}[E(R_{M})-r_{f}]\\
E(R_{i}) & = r_{f}+\beta_{i}[E(R_{M})-r_{f}]
\end{split}
\end{equation}
La \ref{SmL1} viene rappresentata graficamente da una retta che prende il nome di \textit{Security Market Line} (SML):
\begin{figure} [h!]
	\centering
	\includegraphics[width=0.5\linewidth]{"imgs/sml"}
	\caption{Security Market Line \cite{jain_financial_2007}}
	\label{fig:sml}
\end{figure}

La SML è differente rispetto alla CML, quest'ultima descrive il rendimento atteso del portafoglio in funzione della deviazione standard mentre la SML descrive il rendimento atteso di un titolo in funzione del fattore di rischio sistematico $\beta$ e del premio per il rischio. La deviazione standard del titolo include sia il rischio sistematico sia il rischio non sistematico, assumendo che il rischio specifico possa essere azzerato, questo indice non può rappresentare una misura idonea del rischio di un titolo. 

Essendo la Security Market Line una rappresentazione grafica della relazione rendimento-beta i titoli correttamente prezzati giacciono sulla retta. Viceversa i titoli che nel Single Index Model abbiamo definito sovra o sotto prezzati si troveranno rispettivamente sotto o sopra la SML.

\subsection{Validità del CAPM}


Il CAPM rappresenta un valido modello di equilibrio che consente di prezzare i titoli. Grazie alla sua semplicità è stato ed è tutt'ora largamente utilizzato. Vi sono diversi fattori che hanno attirato l'attenzione degli studiosi e che hanno incentivato la diffusione e l'utilizzo di tale modello tra i quali: la scoperta di una diretta relazione fra rendimento e rischio, l'importanza del rischio sistematico nel processo di valutazione del prezzo di un titolo o portafoglio, l'introduzione del fattore $\beta$ come misura del rischio sistematico, ecc. Tuttavia tale modello presenta dei limiti, anzitutto le assunzioni sono molto lontane dalla realtà pensiamo ad esempio all'assenza di tasse o di costi di transazione oppure al fatto che tutti i soggetti hanno accesso a tutte le informazioni disponibili sul mercato senza sostenere dei costi. Inoltre, secondo le affermazioni di Richard Roll, non esiste un portafoglio di mercato reale in quanto gli indici di mercato che vengono utilizzati come proxy non includono la maggior parte della ricchezza degli investitori come ad esempio investimenti stranieri, beni immobili, ecc. Altro problema riguarda il fattore $\beta$, molti economisti fra i quali Fama e French ritengono che vi siano altri fattori che caratterizzano un'impresa che riescono a predire i futuri rendimenti in modo più completo rispetto al solo fattore $\beta$. 


Per fornire una risposta a questo ultimo problema sono stati sviluppati dei modelli definiti multifattoriali, in particolare ne vedremo due: l'Arbitrage Pricing Theory e il Modello a tre fattori di Fama e French. Tali modelli, sulla scia del Capital Asset Pricing Model cercano di spiegare i rendimenti attesi di un titolo ipotizzando diversi fattori di rischio che possono influenzare e quindi spiegare i rendimenti dei titoli.

\section{L'Arbitrage Pricing Theory}
\subsection{Introduzione}


L'\textit{Arbitrage Princing Theory} (APT) elaborata nel 1976 da Stephen Ross rappresenta un modello alternativo al CAPM \cite{ross_arbitrage_2013} in quanto concorda con l'intuizione che il rendimento di un titolo può essere espresso in funzione del rischio. 

Una delle implicazioni del CAPM è la presenza di mercati efficienti e ciò rappresenta proprio il punto di partenza dell'APT. 

Le ipotesi eccessivamente stringenti del CAPM suscitano una forte critica da parte di Stephen Ross il quale esclude l'ipotesi secondo cui i rendimenti seguono una distribuzione di probabilità normale, l'ipotesi di una funzione di utilità quadratica, rifiuta la superiorità del portafoglio di mercato e così via. 

Ciò che viene dimostrato da Ross è la possibilità di costruire un modello di equilibrio che descriva i rendimenti in funzione di diversi fattori di rischio senza assumere l'equilibrio dei mercati ma solo assenza di opportunità di arbitraggio.

Un'opportunità di arbitraggio \cite{dybvig_arbitrage_1989} è una strategia di investimento che garantisce un profitto positivo assumendo un investimento netto pari a zero e assenza di rischio. Un esempio di arbitraggio è dato dall'opportunità di prendere a prestito e prestare a due tassi di interesse differenti. Tale disparità fra il tasso di interesse ricevuto e il tasso pagato non persiste a lungo, le operazioni di arbitraggio infatti porteranno i due tassi di interessi ad un livello di equilibrio.  

Due sono gli elementi che distinguono l'APT dal CAPM \cite{ross_return_1973}: l'APT ammette più di un fattore di rischio e dimostra che poiché l'equilibrio di mercato è coerente con l'assenza di opportunità di arbitraggio, qualsiasi equilibrio di mercato può essere descritto attraverso una relazione lineare fra rendimento e rischio 

Diversi sono gli studi, anche precedenti alla pubblicazione dell'APT, che ipotizzano l'esistenza di un modello multifattoriale \cite{roll_empirical_1980}. 

Farrar e King per esempio si concentrano sull'influenza dell'industria sui rendimenti, Rosenberg e Marathe si focalizzano sull'individuazione delle componenti che influenzano i rendimenti al di fuori del mercato scomponendo il fattore $\beta$ del CAPM in parti differenti. 

E ancora nei propri contributi, Langetieg, Lee, Vinso e Meyers presentano l'idea di un rendimento spiegato attraverso l'influenza di più fattori di rischio diversi dal fattore di rischio di mercato.

\subsection{Ipotesi}
Le assunzioni di partenza dell'APT sono le seguenti:
\begin{itemize}
	\item Il mercato è perfettamente competitivo;
	\item Il rischio specifico è diversificabile;
	\item Il processo di aggiustamento dei prezzi non permette che vi siano opportunità di arbitraggio per un lungo periodo, pertanto se un titolo ha rendimenti maggiori questi saranno dati da un rischio maggiore;
	\item I rendimenti possono essere spiegati utilizzando un modello multifattoriale che include diversi fattori di rischio.
\end{itemize}

\subsection{Portafogli d'arbitraggio}

La principale relazione che emerge dal CAPM è quella che descrive il rendimento di un titolo in funzione di un tasso risk-free, di un fattore di rischio sistematico e del premio per il rischio di mercato: 
\begin{equation}
\label{SmL}
E(R_{i}) = r_{f}+\beta_{i}[E(R_{M})-r_{f}]
\end{equation}
Dove $r_{f}$ è il tasso di interesse risk-free, $[E(R_{M})-r_{f}]$ è il premio per il rischio di mercato e $\beta_{i}=\dfrac{Cov_{i,m}}{\sigma^2_{m}}$ misura la sensitività dei rendimenti del titolo $i$ alle variazioni del mercato. 

L'alternativa proposta dall'Asset Pricing Model è invece la seguente. 

Supponiamo che i rendimenti aleatori dei titoli $i$ siano descritti da un semplice modello multifattoriale costituito da $k$ fattori di rischio aggregato: 
\begin{equation}
\begin{split}
\label{Rendimentoapt2}
\tilde{x}_i & = E_i +\beta_{i1}\tilde{\delta}_i+\beta_{i2}\tilde{\delta}_i+ .......+ \beta_{k1}\tilde{\delta}_k +\tilde{\epsilon} \\
i& = 1,2,3,.....,n
\end{split}
\end{equation}
Dove $E_i$ è il rendimento atteso (ex ante) ed è costante, $\tilde{\delta}_i$ rappresenta il fattore di rischio aggregato, $\beta_i$ è il coefficiente atteso che misura la sensibilità del titolo $i$ al fattore di rischio aggregato infine $\tilde{\epsilon}$ rappresenta il rischio non sistematico. Il modello è stato costruito ipotizzando che la media di $\tilde{\delta}_i$ e di $\tilde{\epsilon}$ siano pari a 0, che il rischio sistematico e non sistematico non siano correlati tra loro $Cov(\tilde{\delta}_i,\epsilon_i)=0$ ed infine che non vi è correlazione fra i rischi idiosincratici di due titoli differenti $Cov(\epsilon_i,\epsilon_j)=0$. 

Assumiamo che il set $n$ di titoli preso in considerazione sia maggiore rispetto al set di $k$ fattori di rischio, tralasciando il rischio non sistematico, il rendimento di un titolo può essere espresso in funzione del tasso risk-free e in funzione dei fattori di rischio, nello stesso modo sia il tasso risk-free che i $k$ fattori possono essere espressi come combinazione lineare di $k+1$ rendimenti. Qualunque rendimento dei titoli, essendo una combinazione lineare dei fattori, potrà essere costruito in funzione dei  $k+1$ rendimenti.

Questo rappresenta il cuore dell'APT: in natura esistono solo poche componenti di rischio sistematico che possono essere: il PIL, il tasso di interesse, ecc.
Ne discende che portafogli costituiti da  $k+1$ titoli sono perfettamente sostituibili e pertanto devono essere prezzati nello stesso modo.

Per sviluppare quest'ultima considerazione partiamo in primo luogo con la costruzione di un portafoglio d'arbitraggio $\eta$ costituito da $n$ titoli. 
\begin{description}
	\item[Step 1.] Tale portafoglio richiede che l'investimento netto sia pari a 0 ciò significa che il capitale utilizzato per acquistare dei titoli deve essere perfettamente controbilanciato dal capitale ottenuto dalla vendita:
	\begin{equation}
	\label{Condizione1}
	\eta e=0
	\end{equation}
	\item[Step 2.]Secondo la legge dei grandi numeri, all'aumentare di $n$, $\tilde{\epsilon}$ tenderà a 0, il rischio specifico può essere ridotto componendo un portafoglio sufficientemente diversificato:
	\begin{equation}
	\eta\tilde{x}_i = \eta E_i+\beta_{i1}\tilde{\delta}_i+\beta_{i2}\tilde{\delta}_i+ .......+ \beta_{k1}\tilde{\delta}_k
	\end{equation}
	\item[Step 3.] Se inoltre richiediamo che non vi sia rischio sistematico 
		\begin{equation}
		\label{Condizione2}
		\begin{split}
		\eta\beta_1 &=	\eta\beta_2 = \eta\beta_3 = ......= 	\eta\beta_k= 0 \\
	\eta\tilde{x}_i&=\eta E_i
		\end{split}
	\end{equation}
	\item [Step 4.] Scegliendo un portafoglio ben diversificato, che comporta un investimento netto pari a 0 e che non richiede l'assunzione di rischio sistematico siamo stati in grado di creare un portafoglio che consenta di ottenere un profitto pari a: $\eta\tilde{x}_i=\eta E_i$. Tuttavia dal momento che tale portafoglio può essere composto senza sostenere alcun costo ne deriva che:
	\begin{equation}
	\label{Noprofitto}
	\eta E_i=0
	\end{equation} 
\end{description}
Se vengono soddisfatte le condizioni \ref{Condizione1} e \ref{Condizione2}, dovrà essere soddisfatta  anche la \ref{Noprofitto}. La formula \ref{Rendimentoapt2} ci consente di dimostrare assenza di opportunità di arbitraggio e di conseguenza equilibrio sul mercato. Pertanto tale formula può fornire un'idonea rappresentazione della relazione che intercorre fra rendimento e rischio in una situazione di equilibrio del mercato. 
 
\`E possibile esprimere la \ref{Noprofitto} attraverso l'algebra lineare: se vi è un vettore che è ortogonale al vettore di valori costanti $\eta E_i$ e al vettore dei $\beta$, quel vettore sarà ortogonale anche al vettore $E$. Dunque $E$ è combinazione lineare del vettore di valori costanti e del rischio sistematico $\beta$. Esistono $k+1$ pesi $\lambda$ tali che: 
\begin{equation}
\label{Rendimento atteo con lambda apt}
E_i=\lambda_0+\lambda_1\beta_{i1}+\lambda_2\beta_{i2}+\lambda_3\beta_{i3}+....+\lambda_k\beta_{ik}
\end{equation}

La figura sotto è una rappresentazione geometrica di quanto affermato fino adesso:

\begin{figure}[h!]
	\centering
	\includegraphics[width=0.6\linewidth]{"imgs/arbitraggio 2"}
	\caption{Opportunità d'arbitraggio}
	\label{fig:arbitraggio-2}
\end{figure}
	

Ipotizziamo di poter costruire dei portafogli con i titoli $1,2,3$, che il loro ammontare sia positivo e che il titolo $2$ sia sopra la retta che congiunge i titoli $1$ e $3$. Il portafoglio costituito dai titoli $1$ e $3$ ha il medesimo rischio del titolo $2$ però offre un rendimento minore. L'arbitraggista potrebbe sfruttare questo scenario vendendo il portafoglio e acquistando il titolo 2, ottenendo un rendimento superiore a parità di rischio sostenuto.

Questa strategia verrà seguita dagli arbitraggisti fino al momento in cui il prezzo del titolo 2 salirà in modo da compensare il rendimento aggiuntivo che offre. 

Affinché non vi siano opportunità di arbitraggio è necessario che $E_{1,3}=E_2$, in altri termini sia il portafoglio che il titolo dovranno giacere sulla stessa retta. 

Se ipotizziamo per esempio un titolo risk-free con rendimento pari a $E_0$ e i fattori di rischio pari a 0 avremo $E_0=\lambda_0$ e potremmo scrivere la \ref{Rendimento atteo con lambda apt} come segue:
\begin{equation}
\label{Rendimentoapt}
E_i-E_0= \lambda_1\beta_{i1}+\lambda_2\beta_{i2}+\lambda_3\beta_{i3}+....+\lambda_k\beta_{ik}
\end{equation}
$E_0$ rappresenta sia il rendimento del titolo risk-free sia il rendimento di qualunque titolo zero-beta.

Arrivati a questo punto risulta fondamentale interpretare il significato di $\lambda$.

Ipotizziamo l'esistenza di un solo fattore di rischio e $\beta=1$ possiamo scrivere il rendimento come segue: 
\begin{equation}
\lambda_1=E_1-E_0 
\end{equation}
Nello stesso modo possiamo ipotizzare un altro portafoglio sensibile al solo fattore $2$, in tal caso avremo: 
\begin{equation}
\lambda_2=E_2-E_0 
\end{equation}
Quindi $\lambda$ rappresenta l'extra rendimento ossia il premio per il rischio di mercato pagato agli investitori che detengono portafogli sensibili ai fattori di rischio aggregato.

Possiamo scrivere la \ref{Rendimentoapt} come segue: 
\begin{equation}
E_i-E_0=(E_1-E_0)\beta_{i1}+(E_2-E_0)\beta_{i2}+........+(E_k-E_0) \beta_{ik}
\end{equation}

Il premio per il rischio di un titolo $i$, ($E_i-E_0$) è la media ponderata dei premi per il rischio generati dai fattori di rischio, i cui pesi sono rappresentati dai coefficienti $\beta$.

Dunque è possibile costruire un modello per la valutazione dei rendimenti di un titolo senza necessariamente ipotizzare un equilibrio sul mercato. Questo risultato deriva infatti dall'assunzione di assenza di opportunità di arbitraggio.

\subsection{Validità dell'APT}
L'APT riesce a riproporre le medesime implicazioni del CAPM ma assumendo ipotesi meno stringenti e senza dover ricorrere al portafoglio di mercato, infatti secondo questa teoria il portafoglio di mercato potrebbe riflettere solo uno dei fattori di rischio aggregato. 

L'utilizzo di poche ipotesi e le implicazioni che derivano da tale modello lo rendono idoneo all'esecuzione di test empirici.

Nonostante ciò una delle principali difficoltà che emerge nell'applicazione di tale teoria risiede nel fatto che non vengono specificati nè la quantità nè la qualità dei fattori di rischio. Inoltre l'applicazione di tale modello richiede l'utilizzo di un'enorme quantità di dati.

Ciò che emerge da studi successivi è l'esistenza di variabili  microeconomiche che influenzano i rendimenti in misura maggiore rispetto alle variabili macroeconomiche. 

Tra questi troviamo per esempio il Modello a tre fattori di Eugene Fama e Kenneth French. 








