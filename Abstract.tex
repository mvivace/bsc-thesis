\begin{abstract}
	
Questo lavoro si propone di tracciare un percorso sulla letteratura in merito ai modelli di pricing fino alla nascita del \textit{Modello a tre fattori} di Fama e French. Verrà illustrata la \textit{Moderna teoria del portafoglio} che ha segnato una svolta rilevante nella finanza introducendo concetti fondamentali come la diversificazione e il criterio media-varianza. L'autore di questa teoria, Henry Markowitz, è ad oggi considerato il padre dell'economia finanziaria poiché ha fornito dei contributi rilevanti che hanno consentito la nascita dei modelli di pricing. Verranno presentati i precursori del Modello a tre fattori: il \textit{Capital Asset Pricing Model} uno dei primi modelli di asset pricing che stabilisce una relazione fra il rendimento atteso di un titolo e il rischio sistematico ad esso associato e l'\textit{Abritrage Pricing Theory} secondo il quale il rendimento di un titolo è funzione lineare di una serie di fattori di rischio. 

Successivamente l'elaborato si focalizzerà sul Modello a tre fattori che descrive il rendimento di un portafoglio in funzione di tre fattori di rischio: il fattore di mercato, il fattore Size e il fattore rappresentato dall'indice Book/Market.

Verranno presentati i risultati degli studi svolti sui rendimenti storici dei portafogli e i risultati dei test eseguiti impiegando tecniche statistiche come la regressione lineare. Infine utilizzando gli stessi strumenti e svolgendo le medesime procedure degli autori verrà effettuata un'analisi dei rendimenti dei portafogli con l'obiettivo di fare emergere una robustezza o debolezza del modello in relazione ai dati relativi all'ultimo decennio. 
	
	
	 

\end{abstract}