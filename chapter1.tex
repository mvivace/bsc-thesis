\chapter{Introduzione}

\section{Motivazioni}
La globalizzazione economica innescata dallo sviluppo tecnologico e dall'apertura dei mercati permette a un gran numero di soggetti ed operatori economici di disporre di un ventaglio molto ampio di possibilità d'investimento e di operare in mercati finanziari di diversi paesi. 

Per poter scegliere la combinazione di investimenti ottimale, che fornisca massimo guadagno e minimo rischio,  gli operatori economici si avvalgono di tecniche e modelli di asset pricing che consentono di prevedere l'andamento futuro dei titoli o dei portafogli.  

Nel corso della storia diversi economisti, statistici e matematici hanno collaborato a lungo allo sviluppo di modelli di pricing sempre più accurati. 

Tali contributi rappresentano oggi un punto di partenza imprescindibile per gli studi futuri.   

Questo lavoro si concentra sul Modello a tre fattori che è stato sviluppato intorno agli anni 90 da Eugene Fama e Kenneth French e si propone di verificare mediante un'analisi descrittiva e di regressione la validità del modello nell'ultimo decennio. I risultati potrebbero evidenziare o meno la necessità di introdurre nuovi modelli che si adattino ai cambiamenti a cui assistiamo ogni giorno anche a causa della globalizzazione.

\section{Struttura}

Per seguire una linea logica e fornire i concetti chiave per la comprensione del modello l'elaborato ripercorre la letteratura in materia di asset pricing evidenziando le tappe cruciali \cite{noauthor_economia_2020}.  
 
Nel secondo capitolo “L'evoluzione dei modelli di pricing” viene presentata la Moderna teoria del portafoglio, il Capital Asset Pricing Model e l'Arbitrage Pricing Theory. In particolare l'autore della Moderna teoria del portafoglio evidenzia la possibilità di ridurre molta parte del rischio attraverso la costruzione di portafogli ben diversificati e la possibilità attraverso l'impiego del criterio media-varianza di discriminare i portafogli efficienti che massimizzano i rendimenti o minimizzano i rischi dai portafogli inefficienti consentendo di fatto di minimizzare le perdite. Queste scoperte rappresenteranno un punto di partenza importante per lo sviluppo dei modelli di pricing e più in generale nell'economia finanziaria.  

Il primo modello di pricing che viene descritto in questo lavoro è invece il Capital Asset Pricing Model, il quale descrive il rendimento di un titolo in funzione del solo rischio sistematico/di mercato, rischio che non può essere diversificato tramite la combinazione di titoli differenti e per il quale gli investitori richiederebbero un premio aggiuntivo.

Ciononostante, si ritiene che le ipotesi siano troppo stringenti e che il singolo fattore sistematico non sia sufficiente a spiegare i rendimenti dei titoli, a questo proposito è stata elaborata una seconda teoria l'Arbitrage Pricing Theory che descrive i rendimenti in funzione di più fattori di rischio assumendo assenza di opportunità di arbitraggio.

Il terzo capitolo è dedicato al modello a tre fattori di Fama e French che in linea con l'APT definisce il rendimento funzione lineare di tre fattori di rischio. Nello specifico i due autori aggiungono al rischio di mercato, il rischio legato alla capitalizzazione di mercato e all'indice Book/Market poiché ritengono che i valori di questi ultimi siano in grado di catturare il rischio associato ad una società. 

Per verificare questa ipotesi effettuano le regressioni cross-section e time-series sui rendimenti dei portafogli dalle quali emerge una connessione tra i fattori e i rendimenti. 

Infine nel quarto capitolo viene presentata l'analisi descrittiva e di regressione dei rendimenti dei portafogli di società appartenenti ai paesi maggiormente sviluppati dal 2010 al 2020. I risultati non rispecchiano quelli ottenuti dagli autori. A riguardo sono state fornite tante spiegazioni tra le quali troviamo per esempio la necessità di introdurre ulteriori fattori di rischio. 

\section{Metodologia e fonti}

Ogni modello è stato contestualizzato, evidenziando le argomentazioni ed i motivi che hanno portato allo sviluppo dello stesso. Sono state fornite le ipotesi o le assunzioni che sostenevano la teoria/il modello e gli strumenti utilizzati. I passaggi matematici sono stati fondamentali per dimostrare le iniziali intuizioni degli autori, infine per ognuno sono stati messi in luce i punti di forza e di debolezza che ne hanno permesso un largo utilizzo oppure un superamento. 

La scrittura di questo elaborato è stata resa possibile da uno studio attento e scrupoloso degli articoli ufficiali pubblicati negli anni dai diversi autori sui “Journal of Financial Economics”, “Journal of Finance” e “Financial Analyst Economics”. 

Mentre i dati necessari per l'analisi sono stati forniti dal sito ufficiale di Kenneth French. Inoltre per effettuare quest'ultima sono state applicate tecniche di machine learning tramite il linguaggio di programmazione Python.






